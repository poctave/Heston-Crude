% ============================================================
%  Dissertation — Spectral Geometry and Calibration Stability
%  of the Heston Model in Crude Oil Options Markets
% ============================================================
\documentclass[12pt, a4paper]{report}

% --- Encoding & font ---
\usepackage[T1]{fontenc}
\usepackage[utf8]{inputenc}
\usepackage{lmodern}

% --- Layout ---
\usepackage[top=2.5cm, bottom=2.5cm, left=2.5cm, right=2.5cm]{geometry}
\usepackage{setspace}
\onehalfspacing

% --- Mathematics ---
\usepackage{amsmath, amssymb, amsthm}

% --- Figures & tables ---
\usepackage{graphicx}
\graphicspath{
  {../../Dissertation/}
  {../../Dissertation/index_plots/}
  {../../Dissertation/results/}
  {../../Dissertation/data_plots/}
}
\usepackage{caption}
\usepackage{subcaption}
\usepackage{booktabs}
\usepackage{float}

% --- Code listings ---
\usepackage{listings}
\usepackage{xcolor}

\definecolor{codebg}{RGB}{248, 248, 248}
\definecolor{codecomment}{RGB}{100, 100, 100}
\definecolor{codekw}{RGB}{0, 100, 180}
\definecolor{codestring}{RGB}{163, 21, 21}
\definecolor{codenum}{RGB}{9, 134, 88}

\lstdefinestyle{python}{
  language=Python,
  backgroundcolor=\color{codebg},
  basicstyle=\ttfamily\footnotesize,
  keywordstyle=\color{codekw}\bfseries,
  stringstyle=\color{codestring},
  commentstyle=\color{codecomment}\itshape,
  numberstyle=\tiny\color{gray},
  numbers=left,
  numbersep=6pt,
  stepnumber=1,
  frame=single,
  framerule=0.5pt,
  rulecolor=\color{gray!40},
  breaklines=true,
  breakatwhitespace=true,
  showstringspaces=false,
  tabsize=4,
  captionpos=b,
  lineskip=-1pt,
  aboveskip=6pt,
  belowskip=4pt,
  xleftmargin=12pt,
  morekeywords={True, False, None, np, pd, scipy}
}
\lstset{style=python}

% --- Multi-page tables and landscape ---
\usepackage{longtable}
\usepackage{pdflscape}
\usepackage{array}

% --- Bibliography ---
\usepackage[round, authoryear]{natbib}

% --- Hyperlinks (load last) ---
\usepackage[colorlinks=true, linkcolor=black, citecolor=blue, urlcolor=blue]{hyperref}

% ============================================================
\begin{document}

% --- Title page ---
\begin{titlepage}
  \centering
  \vspace*{2cm}
  {\LARGE\bfseries Spectral Geometry and Calibration Stability\\
   of the Heston Model\\[0.6em]
   \large A Hessian-Based Analysis of Parameter Identifiability\\
   in Brent and WTI Crude Oil Options\par}
  \vspace{1.2cm}
  {\large\itshape
   What does the spectral geometry of the Heston calibration Hessian reveal\\
   about parameter identifiability in Brent and WTI crude oil options markets,\\
   and across which macro-financial regimes does calibration instability emerge?\par}
  \vspace{1.2cm}
  {\large Octave SUPRANO\par}
  \vspace{0.5cm}
  {\large ESSEC Business School\par}
  \vspace{0.5cm}
  {\large Bachelor in Business Administration\par}
  \vspace{2cm}
  {\large 22 February 2026\par}
  \vfill
  {\normalsize Submitted in partial fulfilment of the requirements for the degree of\\
   Bachelor in Business Administration\par}
\end{titlepage}

% --- Abstract ---
\begin{abstract}
\addcontentsline{toc}{chapter}{Abstract}
This dissertation investigates the calibration stability and parameter identifiability
of the \citet{heston1993} stochastic volatility model when applied to crude oil futures
options. Using daily Bloomberg OVDV implied volatility surfaces for Brent (COA) and WTI
(CLA) from January 2006 to February 2026 — spanning six constant-maturity tenors (1M to
24M) and seven moneyness levels (90\% to 110\% of the forward price) — we construct a
rigorous data pipeline that converts raw Bloomberg exports into a calibration-ready
dataset.

The Heston model is calibrated via a two-stage procedure combining global differential
evolution with local L-BFGS-B refinement, using the numerically stable characteristic
function formulation of \citet{lordkahl2010} and the fast Fourier transform pricer of
\citet{carrmadan1999}. The static analysis snapshot is set to 1 December 2021 — the
most recent date exhibiting the historically dominant crude oil skew regime — yielding
root mean squared errors of 1.54 volatility points for Brent and 2.74 volatility points
for WTI. Both contracts exhibit a negative spot-variance correlation
($\hat{\rho} \approx -0.80$ for Brent and $-0.77$ for WTI), consistent with the
put-skew dominance that characterised oil options markets throughout most of 2006--2022.
Note that since approximately mid-2022, this correlation has persistently inverted to
positive values, reflecting a call-skew regime; the implications of this inversion are
discussed in Chapters~\ref{ch:loading} and~\ref{ch:discussion}. The Feller condition is
violated for both underlyings, indicating that the calibrated variance dynamics can reach
zero --- a structural limitation that motivates the central contribution of this dissertation.

The main contribution is a spectral geometry analysis of the Hessian of the calibration
loss function evaluated at the optimal parameter vector. Decomposing the Hessian into
eigenvalues and eigenvectors reveals which parameter directions are \emph{stiff} (well
constrained by data) and which are \emph{sloppy} (poorly identified), providing a
quantitative characterisation of calibration instability that goes beyond standard
goodness-of-fit metrics. A rolling monthly Hessian analysis over the full 2006--2026
sample yields condition numbers of order $10^5$--$10^6$ for both benchmarks, with the
mean-reversion speed $\kappa$ consistently identified as the sloppy direction.

An empirical chapter links the rolling condition number to five macro-financial
indicators --- OVX, VIX, the Caldara--Iacoviello Geopolitical Risk index, EIA crude
inventories, and the US Dollar Index --- using rule-based regime classification and OLS
regression with Newey--West standard errors. All indicators are negatively correlated
with the condition number: higher market stress produces richer volatility surfaces that
constrain parameters more tightly. The five-indicator model explains 8.9\% and 14.4\%
of condition number variation for Brent and WTI respectively, with geopolitical risk
emerging as the only individually significant predictor for WTI.

A supplementary chapter extends the framework to the \citet{bates1996} Stochastic
Volatility with Jumps (SVJ) model, which augments the Heston dynamics with a compound
Poisson log-normal jump component. The SVJ model is calibrated on the same 1 December
2021 snapshot, and its 8$\times$8 Hessian is subjected to the same spectral decomposition
as the Heston model. This comparison quantifies whether explicitly modelling price jumps
improves parameter identifiability or merely relocates the sloppy subspace.
\end{abstract}

% --- Table of contents ---
\tableofcontents
\listoffigures
\lstlistoflistings

% ============================================================
%  CHAPTER 1 — DATA SOURCES
% ============================================================
\chapter{Data Sources}
\label{ch:data}

This chapter describes the datasets assembled for the calibration study. Four distinct
sources are combined: implied volatility surfaces from the Bloomberg OVDV screen,
front-month crude oil futures prices, the US Treasury yield curve, and a set of
macroeconomic and market indicators used to contextualise the calibration results.
Table~\ref{tab:data_overview} provides a consolidated overview of all series.

\begin{table}[H]
  \centering
  \small
  \caption{Data sources overview}
  \label{tab:data_overview}
  \begin{tabular}{lllll}
    \toprule
    Source                & Series / Ticker        & Coverage            & Freq.   & Purpose              \\
    \midrule
    Bloomberg OVDV        & COA (Brent, 6 tenors)  & Jan.\ 2006--Feb.\ 2026 & Daily & Option calibration   \\
    Bloomberg OVDV        & CLA (WTI, 6 tenors)    & Jan.\ 2006--Feb.\ 2026 & Daily & Option calibration   \\
    Bloomberg             & CO1 Comdty (Brent)     & Jan.\ 2006--Feb.\ 2026 & Daily & Forward / strikes    \\
    Bloomberg             & CL1 Comdty (WTI)       & Jan.\ 2006--Feb.\ 2026 & Daily & Forward / strikes    \\
    Bloomberg             & UST (1M, 3M, 6M, 1Y, 2Y) & Jan.\ 2006--Feb.\ 2026 & Daily & Risk-free rate   \\
    Bloomberg             & OVX Index              & May 2007--Feb.\ 2026   & Daily & Market context       \\
    Bloomberg             & VIX Index              & Jan.\ 2006--Feb.\ 2026 & Daily & Market context       \\
    Bloomberg             & DXY Curncy             & Jan.\ 2006--Feb.\ 2026 & Daily & Market context       \\
    Caldara \& Iacoviello & GPR Index              & Jan.\ 2006--Feb.\ 2026 & Daily & Market context       \\
    EIA                   & US crude inventories   & Jan.\ 2006--present    & Weekly & Market context      \\
    \bottomrule
  \end{tabular}
\end{table}

% -----------------------------------------------------------
\section{Implied Volatility Surfaces}
\label{sec:data_surfaces}
% -----------------------------------------------------------

The primary input to the calibration is the market implied volatility surface, extracted
from the Bloomberg OVDV screen. OVDV is a standardised volatility surface product that
Bloomberg constructs by fitting a smooth surface to listed and OTC option prices; it
is the standard reference for oil options practitioners. For Brent crude, the relevant
ticker is \textbf{COA Comdty} (options on Brent ICE futures); for WTI, it is
\textbf{CLA Comdty} (options on WTI NYMEX futures).

The surface is parameterised in \emph{moneyness} space rather than delta space, which
avoids the need for a delta model and aligns naturally with the calibration framework.
Six constant-maturity tenors are extracted:

\begin{center}
  1 month (1M),\quad 2 months (2M),\quad 3 months (3M),\quad
  6 months (6M),\quad 12 months (12M),\quad 24 months (24M).
\end{center}

At each tenor, implied volatilities are reported at seven moneyness levels defined
relative to the contemporaneous forward price $F$:
\begin{equation}
  m_j \in \{90\%,\; 95\%,\; 97.5\%,\; 100\%,\; 102.5\%,\; 105\%,\; 110\%\}
  \label{eq:moneyness_grid}
\end{equation}
The full grid therefore contains $6 \times 7 = 42$ implied volatility points per
underlying per day. Bloomberg delivers the data as a BDH export with one Excel sheet
per tenor, values expressed as annualised implied volatilities in percent (e.g.\ 32.81
means $32.81\%$ per annum).

Data were downloaded using the Bloomberg Excel \texttt{BDH} (Bloomberg Data History)
function, which retrieves a historical time series for a given security and set of
fields. The query for the 1-month Brent surface takes the form:

\begin{lstlisting}[language={}, basicstyle=\ttfamily\scriptsize,
                   lineskip=-2pt, breaklines=true,
                   caption={Bloomberg BDH query for the 1-month Brent implied
                            volatility surface (Excel formula).},
                   label={lst:bdh}]
=BDH("COA Comdty",
     "1ST_MTH_IMPVOL_80%MNY_DF,
      1ST_MTH_IMPVOL_90.0%MNY_DF,
      1ST_MTH_IMPVOL_95.0%MNY_DF,
      1ST_MTH_IMPVOL_97.5%MNY_DF,
      1ST_MTH_IMPVOL_100.0%MNY_DF,
      1ST_MTH_IMPVOL_102.5%MNY_DF,
      1ST_MTH_IMPVOL_105.0%MNY_DF,
      1ST_MTH_IMPVOL_110.0%MNY_DF,
      1ST_MTH_IMPVOL_120%MNY_DF",
     "01/01/2006", "02/19/2026")
\end{lstlisting}

The field prefix \texttt{1ST\_MTH} identifies the front-month (1M) tenor; equivalent
queries for the remaining tenors substitute \texttt{2ND\_MTH}, \texttt{3RD\_MTH},
\texttt{6TH\_MTH}, \texttt{12TH\_MTH}, and \texttt{24TH\_MTH} respectively.
For WTI the security identifier is replaced by \texttt{CLA Comdty}; all field names
remain identical.
The raw pull retrieves \emph{nine} moneyness levels per tenor (80\%--120\%); the
outermost wings (80\% and 120\%) are excluded during quality filtering
(Chapter~\ref{ch:loading}), leaving the seven-point grid of
equation~\eqref{eq:moneyness_grid}.

The sample runs from 3 January 2006 to 19 February 2026, yielding approximately
4,830 trading days for Brent (COA) and 5,030 for WTI (CLA). With 42 implied
volatility points per day (6 tenors $\times$ 7 moneyness levels), the raw surface data
comprises \textbf{198,470 observations for Brent} and \textbf{210,385 for WTI} prior to
any quality filtering. Table~\ref{tab:surface_summary} summarises the surface data.

\begin{table}[H]
  \centering
  \caption{Implied volatility surface data summary}
  \label{tab:surface_summary}
  \begin{tabular}{lrrrrrr}
    \toprule
    Underlying & Bloomberg & Tenors & Moneyness & Obs./day & Trading days & Total obs. \\
    \midrule
    Brent (CO1) & COA Comdty & 6 (1M–24M) & 7 (90\%–110\%) & 42 & $\approx 4{,}830$ & $198{,}470$ \\
    WTI   (CL1) & CLA Comdty & 6 (1M–24M) & 7 (90\%–110\%) & 42 & $\approx 5{,}030$ & $210{,}385$ \\
    \bottomrule
  \end{tabular}
\end{table}

Two stylised facts characterise the crude oil volatility surface and are illustrated in
Figure~\ref{fig:atm_structure}. First, the ATM term structure is strongly downward
sloping: short-dated implied volatility (around 45--48\% at 1M) is substantially higher
than long-dated volatility (around 24--25\% at 24M), reflecting the mean-reverting
nature of oil prices. Second, the 1-month moneyness smile is \emph{positively} skewed —
implied volatility increases from left to right across the moneyness grid — indicating
that the market assigns a premium to OTM calls relative to OTM puts. This upside
premium reflects the market's expectation of sudden supply-shock driven price spikes
(geopolitical events, OPEC decisions), a feature absent in equity markets where the
leverage effect creates the opposite pattern.

\begin{figure}[H]
  \centering
  \includegraphics[width=\textwidth]{atm_term_structure.png}
  \caption{Market implied volatility structure on 19 February 2026.
           \textit{Left:} ATM term structure (moneyness = 100\%) for Brent (CO1, seagreen)
           and WTI (CL1, steelblue). \textit{Right:} 1-month moneyness smile for both
           underlyings. Source: Bloomberg OVDV.}
  \label{fig:atm_structure}
\end{figure}

% -----------------------------------------------------------
\section{Crude Oil Futures Prices}
\label{sec:data_futures}
% -----------------------------------------------------------

The calibration requires knowing the contemporaneous forward price $F$ for each option.
We use the front-month futures contract as a proxy: \textbf{CO1 Comdty} (Brent ICE
front-month) and \textbf{CL1 Comdty} (WTI NYMEX front-month). These are continuous
rolling contracts constructed by Bloomberg that splice successive front-month settlements.
Daily settlement prices are collected over the same sample period as the vol surfaces,
yielding approximately 5,200 observations per series.

\begin{table}[H]
  \centering
  \caption{Crude oil futures price data summary}
  \label{tab:futures_summary}
  \begin{tabular}{lllll}
    \toprule
    Contract & Underlying  & Exchange & Sample              & Observations \\
    \midrule
    CO1 Comdty & Brent crude & ICE (London)   & Jan.\ 2006--Feb.\ 2026 & $\approx 5{,}200$ \\
    CL1 Comdty & WTI crude   & NYMEX (New York) & Jan.\ 2006--Feb.\ 2026 & $\approx 5{,}200$ \\
    \bottomrule
  \end{tabular}
\end{table}

The futures price serves a dual role in the pipeline: it provides the forward price $F$
needed to convert moneyness levels into absolute strikes ($K = m \times F$), and it
enters the Black-76 option pricing formula as the underlying. Figure~\ref{fig:price_evolution}
displays the full price history. The two benchmarks co-move closely, with Brent
typically at a modest premium to WTI. Key episodes visible in the series include the
2008 financial crisis (prices fell from a record \$147 to below \$40), the 2014--2016
supply glut (Saudi Arabia maintained output despite falling prices), the COVID-19 crash
of March 2020 (WTI briefly traded negative on 20 April 2020 due to storage constraints),
and the 2022 energy shock following the Russian invasion of Ukraine.

\begin{figure}[H]
  \centering
  \includegraphics[width=\textwidth]{price_evolution.png}
  \caption{Brent (CO1) and WTI (CL1) front-month futures prices, VIX and OVX volatility
           indices, January 2006 -- February 2026. Source: Bloomberg.}
  \label{fig:price_evolution}
\end{figure}

% -----------------------------------------------------------
\section{Risk-Free Rate}
\label{sec:data_rfr}
% -----------------------------------------------------------

Options on futures require discounting at the risk-free rate $r$ over the option's
remaining maturity. We use US Treasury constant-maturity yields obtained from Bloomberg,
which provide a daily market-implied curve. Five tenors are extracted to span the
maturity range of the option data:
\begin{center}
  1 month,\quad 3 months,\quad 6 months,\quad 1 year,\quad 2 years.
\end{center}

Bloomberg delivers multi-series BDH exports in an interleaved format, with each tenor
occupying a separate pair of columns (date column followed by yield column). For option
maturities that fall between two extracted tenors (e.g.\ the 2-month option maturity
falls between the 1-month and 3-month tenors), the risk-free rate is obtained by linear
interpolation across the yield curve. This ensures that each option observation is
discounted at a maturity-matched rate rather than a single scalar.

\begin{table}[H]
  \centering
  \caption{US Treasury yield curve data}
  \label{tab:ust_summary}
  \begin{tabular}{llll}
    \toprule
    Tenor & Bloomberg ticker & Maturity in years & Interpolated to option tenors \\
    \midrule
    1 month  & USGG1M Index & $1/12$  & 1M option \\
    3 months & USGG3M Index & $3/12$  & 2M, 3M options (direct + interpolation) \\
    6 months & USGG6M Index & $6/12$  & 6M option \\
    1 year   & USGG12M Index & $1.0$ & 12M option \\
    2 years  & USGG2Y Index & $2.0$  & 24M option \\
    \bottomrule
  \end{tabular}
\end{table}

% -----------------------------------------------------------
\section{Macroeconomic and Market Indicators}
\label{sec:data_macro}
% -----------------------------------------------------------

To provide broader market context, we collect five additional series:

\begin{itemize}
  \item \textbf{OVX} (CBOE Crude Oil Volatility Index): the VIX equivalent for WTI
        crude oil, computed from 30-day ATM options on the United States Oil Fund (USO).
        Available from 10 May 2007, OVX provides a real-time market expectation of
        30-day oil price volatility and serves as a natural benchmark for the ATM implied
        volatility extracted from the OVDV surface.
  \item \textbf{VIX} (CBOE Volatility Index): the standard measure of equity market
        implied volatility, included to capture cross-asset risk-off episodes that
        spill over into commodity markets.
  \item \textbf{DXY} (ICE US Dollar Index): since crude oil is globally priced in US
        dollars, a stronger dollar tends to suppress oil demand from non-dollar economies,
        creating a systematic negative correlation between DXY and oil prices.
  \item \textbf{Geopolitical Risk Index}: the daily index of \citet{caldara2022},
        constructed from automated text searches of major newspaper archives. Spikes
        correspond to events such as the 2003 Iraq War, the 2011 Arab Spring, and the
        2022 Russia-Ukraine conflict.
  \item \textbf{US crude oil inventories} (EIA, weekly): total commercial crude stocks
        reported each Wednesday by the US Energy Information Administration. Both the
        level and the weekly change are stored, as unexpected inventory builds or draws
        are primary short-term drivers of oil price volatility.
\end{itemize}

\begin{figure}[H]
  \centering
  \includegraphics[width=\textwidth]{macro_indicators.png}
  \caption{Macroeconomic and oil market indicators: OVX (top), geopolitical risk index
           of \citet{caldara2022} (second), weekly oil inventory changes (third), and
           total US crude oil inventory level (bottom).
           Source: Bloomberg; \citet{caldara2022}; EIA.}
  \label{fig:macro_indicators}
\end{figure}

% ============================================================
%  CHAPTER 2 — DATA LOADING AND CLEANING
% ============================================================
\chapter{Data Loading and Cleaning}
\label{ch:loading}

This chapter details the data pipeline implemented in \texttt{surface\_loader.py}.
Starting from raw Bloomberg Excel exports, the pipeline performs five sequential
operations: loading and parsing the vol surface sheets, loading futures prices,
computing strikes from moneyness, interpolating the risk-free rate, and assembling
the final calibration-ready dataset. The output is a \texttt{pandas} DataFrame of
407{,}679 rows covering both underlyings from 4 January 2006 to 19 February 2026,
with no missing values in any of the nine required fields.

% -----------------------------------------------------------
\section{Loading the Implied Volatility Surfaces}
\label{sec:loading_surfaces}
% -----------------------------------------------------------

Each Bloomberg OVDV export consists of one Excel workbook with six sheets (one per
tenor). Within each sheet, the first row is a Bloomberg metadata header (ticker and
field name) that is discarded; the second row contains column labels (a \texttt{Date}
label followed by seven moneyness levels); and data begins on the third row. Because
Bloomberg may represent the 97.5\% and 102.5\% moneyness columns as strings
(\texttt{"97.5\%"}) or as floating-point numbers ($0.975$ or $97.5$), a normalisation
helper is applied to unify all representations to a decimal fraction in $[0, 2]$:

\begin{lstlisting}[caption={Moneyness column header normalisation
                   (\texttt{surface\_loader.py})},
                   label={lst:parse_moneyness}]
def _parse_moneyness(val) -> float:
    """Normalise a moneyness header to a decimal fraction.
    Handles: 0.975, '0.975', '97.5%', 97.5, '97.5'
    """
    try:
        s = str(val).strip().replace('%', '').replace('\xa0', '')
        f = float(s)
        return f / 100 if f > 5 else f  # '97.5' -> 0.975; '0.975' -> 0.975
    except (ValueError, TypeError):
        return np.nan
\end{lstlisting}

The surface loading function opens the workbook once via \texttt{pd.ExcelFile} to avoid
repeatedly reading the file from disk, then iterates over all six tenor sheets. For
each sheet, a \texttt{dropna(how=`all')} call removes the all-\texttt{NaN} Bloomberg
metadata row, leaving the column-header row as \texttt{iloc[0]} and the data as
\texttt{iloc[1:]}. Implied volatility values are converted from percent to decimal by
dividing by 100:

\begin{lstlisting}[caption={Loading and reshaping one vol surface Excel file into
                   long-format (\texttt{surface\_loader.py})},
                   label={lst:load_surface}]
def load_vol_surface(filepath, underlying_code):
    prefix = "COA" if underlying_code == "CO1" else "CLA"
    xls = pd.ExcelFile(filepath)

    frames = []
    for tenor, sheet in zip(TENORS, SHEET_NAMES):
        raw  = pd.read_excel(xls, sheet_name=sheet, header=None)
        raw  = raw.dropna(how='all').reset_index(drop=True)
        # Row 0: column headers; Rows 1+: data
        header_row = raw.iloc[0]
        data       = raw.iloc[1:].reset_index(drop=True)

        dates = pd.to_datetime(data.iloc[:, 1], errors='coerce').dt.normalize()

        for ci in range(2, min(9, raw.shape[1])):
            m = _parse_moneyness(header_row.iloc[ci])
            if np.isnan(m) or not (0.5 <= m <= 2.0):
                continue
            ivol_dec = pd.to_numeric(data.iloc[:, ci], errors='coerce') / 100
            frames.append(pd.DataFrame({
                "Date": dates, "Underlying": underlying_code,
                "Maturity": TENOR_MAP[tenor],
                "Moneyness": round(m, 4), "ImpliedVol": ivol_dec,
            }))

    df = pd.concat(frames, ignore_index=True)
    return df.dropna(subset=["Date", "ImpliedVol"]).sort_values(
        ["Date", "Maturity", "Moneyness"]
    ).reset_index(drop=True)
\end{lstlisting}

The function returns a long-format DataFrame with five columns:
\texttt{Date}, \texttt{Underlying}, \texttt{Maturity} (in years),
\texttt{Moneyness}, and \texttt{ImpliedVol} (in decimal form).
After applying this function to both the COA (Brent) and CLA (WTI) files, the combined
surface DataFrame contains 408{,}855 rows before the subsequent merging steps.

Figures~\ref{fig:surface_brent} and~\ref{fig:surface_wti} display the full market
implied volatility surfaces for Brent and WTI on the most recent available date,
19 February 2026. The surfaces exhibit the positive moneyness skew and downward
term structure discussed in Chapter~\ref{ch:data}, and it is the task of the Heston
model (Chapter~\ref{ch:methodology}) to fit this structure with five parameters.

\begin{figure}[H]
  \centering
  \includegraphics[width=0.9\textwidth]{market_surface_brent.png}
  \caption{Bloomberg market implied volatility surface for Brent crude (CO1) on
           19 February 2026. Coloured mesh: OVDV surface interpolated over the
           $6 \times 7$ grid. Red points: observed market data.
           Source: Bloomberg OVDV (COA Comdty).}
  \label{fig:surface_brent}
\end{figure}

\begin{figure}[H]
  \centering
  \includegraphics[width=0.9\textwidth]{market_surface_wti.png}
  \caption{Bloomberg market implied volatility surface for WTI crude (CL1) on
           19 February 2026. Same conventions as Figure~\ref{fig:surface_brent}.
           Source: Bloomberg OVDV (CLA Comdty).}
  \label{fig:surface_wti}
\end{figure}

% -----------------------------------------------------------
\section{Loading Futures Prices and Computing Strikes}
\label{sec:loading_futures}
% -----------------------------------------------------------

The \texttt{Data-Hardcodded.xlsx} workbook contains, among other series, the daily
settlement prices for CO1 and CL1 on separate sheets. Each sheet follows the standard
Bloomberg BDH single-series layout: one empty metadata row, one header row, and data
from the third row onward, with dates in column B and prices in column C.

\begin{lstlisting}[caption={Loading front-month futures prices from
                   \texttt{Data-Hardcodded.xlsx} (\texttt{surface\_loader.py})},
                   label={lst:load_futures}]
def load_new_futures(filepath):
    def _load_sheet(sheet_name, col_label):
        raw  = pd.read_excel(filepath, sheet_name=sheet_name, header=None)
        data = raw.dropna(how='all').iloc[2:].reset_index(drop=True)
        sub  = pd.DataFrame({
            "Date":    pd.to_datetime(data.iloc[:, 1], errors='coerce'),
            col_label: pd.to_numeric(data.iloc[:, 2], errors='coerce'),
        }).dropna()
        return sub[sub["Date"] >= "1980-01-01"]   # drop corrupt serial-date rows

    brent = _load_sheet("Brent Futures", "CO1")
    wti   = _load_sheet("WTI Futures",   "CL1")
    return brent.merge(wti, on="Date", how="outer").sort_values("Date")
\end{lstlisting}

The \texttt{>= 1980-01-01} filter silently drops a known corrupt entry in the WTI sheet
where a Bloomberg export artefact encoded a date as Excel serial number 67
(corresponding to 7 March 1900). Because the vol surface data begins only in January
2006, the filter has no effect on the usable data range.

Once futures prices are available, strikes are computed by a simple merge and
multiplication:
\begin{equation}
  K_{i} = m_j \times F_t
  \label{eq:strike_calc}
\end{equation}
where $m_j$ is the moneyness level from the OVDV grid \eqref{eq:moneyness_grid} and
$F_t$ is the front-month futures settlement price on date $t$. This operation converts
the moneyness-quoted vol surface into an absolute-strike vol surface, which is required
by the Black-76 and Heston pricers. On days where a vol surface observation exists but
no futures settlement is available (typically non-trading days in one market), the row
is silently dropped; this affects 1{,}176 observations ($0.3\%$ of raw surface rows).

% -----------------------------------------------------------
\section{Risk-Free Rate Interpolation}
\label{sec:loading_rfr}
% -----------------------------------------------------------

The Bloomberg BDH export for the yield curve is unusual in structure: when multiple
series are requested in a single query, Bloomberg writes them side by side in an
interleaved pattern, with each tenor occupying a date column followed by a yield column,
separated by an empty column. The five tenors (1M, 3M, 6M, 1Y, 2Y) therefore occupy
column pairs at 0-based indices $(1,2)$, $(4,5)$, $(7,8)$, $(10,11)$, and $(13,14)$
of the exported worksheet:

\begin{lstlisting}[caption={Parsing the interleaved Bloomberg UST multi-series export
                   (\texttt{surface\_loader.py})},
                   label={lst:load_ust}]
UST_TENORS    = ["1M", "3M", "6M", "1Y", "2Y"]
UST_COL_PAIRS = [(1, 2), (4, 5), (7, 8), (10, 11), (13, 14)]

def load_ust_curve(filepath):
    raw  = pd.read_excel(filepath, sheet_name="USTs", header=None)
    data = raw.dropna(how='all').iloc[2:].reset_index(drop=True)
    frames = []
    for tenor, (date_col, val_col) in zip(UST_TENORS, UST_COL_PAIRS):
        dates = pd.to_datetime(data.iloc[:, date_col], errors='coerce')
        rates = pd.to_numeric(data.iloc[:, val_col],   errors='coerce') / 100
        sub   = pd.DataFrame({"Date": dates, "tenor": tenor, "rate": rates})
        frames.append(sub.dropna())
    return pd.concat(frames, ignore_index=True).sort_values(["Date", "tenor"])
\end{lstlisting}

For each option maturity $T$ (in years), the risk-free rate is obtained by linear
interpolation across the five extracted tenor points:

\begin{lstlisting}[caption={Linear interpolation of the UST yield curve to
                   option maturities (\texttt{surface\_loader.py})},
                   label={lst:interp_rfr}]
def _interp_rfr_for_maturity(pivot, t_years, mat_years):
    """Return a Series of interpolated rates indexed by Date."""
    rates  = pivot.values          # shape (n_dates, n_tenors)
    result = np.array([np.interp(mat_years, t_years, row) for row in rates])
    return pd.Series(result, index=pivot.index)
\end{lstlisting}

Maturities at or beyond the 2-year boundary are clamped to the 2-year rate (the
\texttt{np.interp} default). On the 5{,}775 dates where no yield curve observation
is available but a vol surface observation exists (public holidays and other
non-trading days in the Treasury market), the risk-free rate is imputed by
forward-filling the most recently observed rate within each (underlying, maturity) group,
followed by backward-fill for any remaining gaps at the start of the sample. After this
imputation, no missing risk-free rate observations remain in the final dataset.

% -----------------------------------------------------------
\section{Final Dataset}
\label{sec:final_dataset}
% -----------------------------------------------------------

The five processing steps — surface loading, futures loading, strike calculation,
UST loading, and RFR interpolation — are orchestrated by the \texttt{build\_options\_df}
function, which assembles and returns the calibration-ready DataFrame in a single call:

\begin{lstlisting}[caption={Top-level data assembly function
                   (\texttt{surface\_loader.py})},
                   label={lst:build_df}]
def build_options_df(coa_surface_file=COA_FILE,
                     cla_surface_file=CLA_FILE,
                     macro_file=MACRO_FILE,
                     start_date=None, end_date=None):
    # 1. Load vol surfaces -> long format
    coa_df = load_vol_surface(coa_surface_file, "CO1")
    cla_df = load_vol_surface(cla_surface_file, "CL1")
    surface_df = pd.concat([coa_df, cla_df], ignore_index=True)

    # 2. Merge futures: Strike = Moneyness * SpotPrice
    futures_df = load_new_futures(macro_file)
    surface_df = surface_df.merge(fut_long, on=["Date", "Underlying"], how="left")
    surface_df["Strike"] = surface_df["Moneyness"] * surface_df["SpotPrice"]

    # 3. Interpolate RFR for each (Date, Maturity) pair
    ust_df = load_ust_curve(macro_file)
    pivot, t_years = _build_rfr_pivot(ust_df)
    for mat in surface_df["Maturity"].unique():
        rfr_series = _interp_rfr_for_maturity(pivot, t_years, mat)
        # ... merge back

    surface_df["OptionType"]  = "C"
    surface_df["OptionPrice"] = np.nan
    return surface_df[OUT_COLS].dropna(subset=["Strike", "SpotPrice"])
\end{lstlisting}

After assembly and cleaning, the final dataset comprises \textbf{407{,}679 rows}
covering \textbf{both underlyings} (CO1 and CL1), \textbf{six maturities}
($1/12$, $2/12$, $3/12$, $1/2$, $1$, and $2$ years), \textbf{seven moneyness levels}
(90\% to 110\%), and \textbf{approximately 4{,}870 trading days} from 4 January 2006
to 19 February 2026. The nine required fields — \texttt{Date}, \texttt{Underlying},
\texttt{Strike}, \texttt{Maturity}, \texttt{OptionType}, \texttt{ImpliedVol},
\texttt{OptionPrice}, \texttt{SpotPrice}, and \texttt{RiskFreeRate} — contain no
missing values after the imputation steps described above.

Figure~\ref{fig:atm_history} shows the time series of 1-month ATM implied volatility
for both underlyings over the full sample. The series exhibits the characteristic
features of energy market volatility: long stretches of moderate volatility (20--40\%)
punctuated by sharp spikes during the 2008 financial crisis (up to 120\%), the 2014
supply glut (elevated for an extended period), the COVID-19 shock of March 2020
(briefly exceeding 150\% for WTI, reflecting the extreme dislocation in the futures
market when front-month WTI traded at negative prices), and the 2022 Ukraine-related
energy shock.

\begin{figure}[H]
  \centering
  \includegraphics[width=\textwidth]{atm_vol_history.png}
  \caption{1-month ATM ($m = 100\%$) implied volatility for Brent (CO1, seagreen) and
           WTI (CL1, steelblue), January 2006 -- February 2026. Dashed vertical lines
           mark key market events. Source: Bloomberg OVDV.}
  \label{fig:atm_history}
\end{figure}

% ============================================================
%  SECTION 2.6 — SMILE STYLISED FACTS
% ============================================================
\section{Smile Stylised Facts}
\label{sec:smile_stylised}

The assembled dataset reveals three robust empirical regularities in the implied
volatility surface of crude oil options. These stylised facts motivate both the
choice of the Heston stochastic volatility model and the subsequent Hessian
spectral analysis of its calibration landscape.

% --- 2.6.1 Average Smile Shape ---
\subsection{Average Smile Shape}

Figure~\ref{fig:avg_smile_shape} shows the time-averaged implied volatility smile
expressed as a deviation from the at-the-money level,
$\Delta\sigma(m) = \sigma(m) - \sigma_{\mathrm{ATM}}$,
for each of the six constant-maturity tenors. Shaded bands denote $\pm 1$ standard
deviation across trading days. Across both Brent and WTI, the smile is
systematically asymmetric: the right wing (out-of-the-money calls, $m > 1$) is
consistently elevated above ATM, while the left wing (out-of-the-money puts,
$m < 1$) sits below it. This \emph{positive skew} — the structural opposite of the
leverage effect documented in equity markets — reflects the market's persistent
premium for upside optionality in crude oil. Geopolitical supply disruptions and
OPEC production decisions generate non-trivial probability mass for large upside
price moves, inducing higher demand for OTM calls than OTM puts. This finding is
consistent with the positive spot-variance correlation $\hat{\rho} \approx +0.77$
(Brent) and $\hat{\rho} \approx +0.81$ (WTI) recovered in Chapter~\ref{ch:methodology}.
The positive skew is most pronounced at short maturities (1M) and attenuates at
longer tenors, consistent with mean-reversion damping transient supply shocks over
time.

\begin{figure}[H]
  \centering
  \includegraphics[width=\textwidth]{avg_smile_shape.png}
  \caption{Time-averaged implied volatility smile shape for Brent (CO1, left) and
           WTI (CL1, right), 2006--2026. Each line shows
           $\Delta\sigma(m) = \sigma(m) - \sigma_{\mathrm{ATM}}$ averaged over all
           trading days for a given maturity; shaded bands indicate $\pm 1$ standard
           deviation. The systematic elevation of the right wing relative to the
           left wing reflects crude oil's structural positive skew (upside call
           premium). Source: Bloomberg OVDV.}
  \label{fig:avg_smile_shape}
\end{figure}

% --- 2.6.2 Crisis-Period Smile Distortions ---
\subsection{Crisis-Period Smile Distortions}

Figure~\ref{fig:crisis_smile} illustrates how the smile deforms across all six
maturities during selected market stress episodes, relative to a calm-period
baseline (June--August 2012). Each panel in the $2 \times 6$ grid shows
$\Delta\sigma(m)$ for a given underlying and maturity, with five overlaid curves
representing the five market regimes.

Three qualitatively distinct deformation patterns emerge. During the
\textbf{Global Financial Crisis} (October 2008), demand-side collapse risk
elevated the left wing across all maturities, steepening the smile while broadly
inflating the level of uncertainty. The \textbf{Supply Glut} of January 2016
produced a strong left-wing spike at short maturities as the market priced the
risk of further steep price declines in an oversupplied market. The
\textbf{COVID-19} shock of April 2020 — particularly stark given the
extraordinary negative settlement of WTI front-month futures on 20 April 2020 —
produced the sharpest left-wing distortion in the sample, as put demand surged on
fears of physical storage constraints and demand destruction. Conversely, the
\textbf{Ukraine} shock of March 2022 amplified the right wing across short
maturities, reflecting a supply-disruption premium and elevated probability of
further upside price moves from energy sanctions.

\begin{figure}[H]
  \centering
  \includegraphics[width=\textwidth]{crisis_smile_distortion.png}
  \caption{Implied volatility smile deformation,
           $\Delta\sigma(m) = \sigma(m) - \sigma_{\mathrm{ATM}}$ [vol pts],
           across all six maturities (columns) for Brent CO1 (top row) and
           WTI CL1 (bottom row) during four market stress episodes and a
           calm baseline (June--August 2012). Source: Bloomberg OVDV.}
  \label{fig:crisis_smile}
\end{figure}

% --- 2.6.3 Risk Reversal Dynamics ---
\subsection{Risk Reversal Dynamics}

Figure~\ref{fig:risk_reversal} plots the 1M and 3M risk reversal proxy
$\mathrm{RR}(m) = \sigma(110\%) - \sigma(90\%)$ over the full 2006--2026 sample.
Positive values indicate that equidistant OTM calls are more expensive than puts
(upside premium), while negative values signal that downside protection dominates.

Under normal conditions both underlyings exhibit a persistently positive risk
reversal, confirming the structural upside premium identified in the average smile
analysis. The 1M reversal is more volatile than the 3M, reflecting the larger
impact of near-term event risk on the short end of the term structure. Two
prominent sign reversals are visible: the COVID-19 episode (March--May 2020) drove
both risk reversals sharply negative as put demand overwhelmed call buying, and
the supply-glut period (2015--2016) produced a sustained period of near-zero or
negative reversal for WTI. The Ukraine war spike (2022) restored strongly positive
reversals as supply-side uncertainty reignited call premiums.

This time-varying skew direction — flipping sign between supply- and
demand-driven crises — presents a fundamental challenge for any single-regime
stochastic volatility model: the Heston model must fit qualitatively different
smile shapes (positive and negative skew) with a fixed parameter vector, which
manifests as poor local identifiability in the spectral analysis of the
calibration Hessian presented in Chapter~\ref{ch:hessian}.

\begin{figure}[H]
  \centering
  \includegraphics[width=\textwidth]{risk_reversal_ts.png}
  \caption{1M (solid) and 3M (dashed) risk reversal proxy
           $\sigma(110\%) - \sigma(90\%)$ [vol pts] for Brent CO1 (top) and
           WTI CL1 (bottom), 2006--2026. Positive values indicate a call
           premium (upside risk); negative values indicate a put premium
           (downside fear). Annotated events as in Figure~\ref{fig:atm_history}.
           Source: Bloomberg OVDV.}
  \label{fig:risk_reversal}
\end{figure}

% ============================================================
%  CHAPTER 3 — METHODOLOGY
% ============================================================
\chapter{Methodology}
\label{ch:methodology}

\section{The Heston Stochastic Volatility Model}

We model crude oil futures prices under the risk-neutral measure using the stochastic
volatility framework of \citet{heston1993}. Let $S_t$ denote the futures price and
$V_t$ its instantaneous variance. The joint dynamics are:

\begin{align}
  dS_t &= \mu S_t\,dt + \sqrt{V_t}\,S_t\,dW_t^1 \label{eq:heston_S} \\
  dV_t &= \kappa(\theta - V_t)\,dt + \sigma\sqrt{V_t}\,dW_t^2 \label{eq:heston_V}
\end{align}

where $dW_t^1$ and $dW_t^2$ are correlated Brownian motions with
$\mathrm{d}W_t^1\,\mathrm{d}W_t^2 = \rho\,dt$. The model has five parameters,
described in Table~\ref{tab:heston_params}.

\begin{table}[H]
  \centering
  \caption{Heston model parameters}
  \label{tab:heston_params}
  \begin{tabular}{clll}
    \toprule
    Parameter & Symbol & Economic interpretation & Calibration bounds \\
    \midrule
    Mean-reversion speed   & $\kappa$ & Speed at which variance reverts to $\theta$ & $[0.01,\;20]$       \\
    Long-run variance      & $\theta$ & Unconditional variance; $\sqrt{\theta}$ = long-run vol & $[0.001,\;2]$ \\
    Vol of vol             & $\sigma$ & Volatility of the variance process          & $[0.01,\;2]$        \\
    Spot-variance corr.    & $\rho$   & Leverage effect; typically negative in equities & $[-0.999,\;0.999]$ \\
    Initial variance       & $v_0$    & Variance at calibration date; $\sqrt{v_0}$ = initial vol & $[0.001,\;2]$ \\
    \bottomrule
  \end{tabular}
\end{table}

\subsection{Feller Condition}

For the variance process \eqref{eq:heston_V} to remain strictly positive almost surely,
the following condition must hold:
\begin{equation}
  2\kappa\theta \geq \sigma^2
  \label{eq:feller}
\end{equation}
This is the Feller condition. In practice, we do not impose it as a hard constraint
during optimisation, as doing so can prevent the algorithm from reaching the global
minimum. Instead, we report whether it is satisfied as a diagnostic after calibration.

\subsection{Crude Oil Context: The Sign of Correlation}
\label{sec:heston_oil}

In equity markets the spot-variance correlation $\rho$ is typically negative
(the \emph{leverage effect}): falling prices are accompanied by rising volatility.
In crude oil markets, the historical norm has been the opposite.
Over most of the 2006--2022 sample, the calibrated $\rho$ is negative
($\hat{\rho} \approx -0.80$ for Brent and $-0.77$ for WTI on the static date of
1 December 2021), reflecting \emph{put-skew dominance}: out-of-the-money puts carry
higher implied volatility than equidistant calls. This pattern is consistent with
the demand-crash risk premium \citep{schwartz1997, trolle2009} — large downward
price moves driven by demand collapses (as during COVID-19) or inventory gluts
are the dominant tail risk for oil markets over the long run.

Formally, a negative $\rho$ tilts the risk-neutral log-price distribution
leftward, generating a downward-sloping smile (put skew). This is the stylised
fact documented in Figure~\ref{fig:avg_smile_shape} for the time-averaged surface.

\paragraph{The post-2022 call-skew inversion.}
Since approximately mid-2022, the rolling calibration has consistently produced
positive $\rho$ values (up to $+0.77$ for Brent and $+0.81$ for WTI as of
early 2026), reflecting a persistent \emph{call-skew regime}. This inversion
coincides with the Ukraine war supply shock and the subsequent structural
tightening of global oil supply. Supply-driven upside risk — where price spikes
and volatility co-move positively — generates the inverse leverage effect
documented in commodity markets \citep{schwartz1997, trolle2009}.
The rolling risk-reversal time series in Figure~\ref{fig:risk_reversal} shows
this sign change clearly. Because the static analysis snapshot (1 December 2021)
pre-dates this inversion, the calibrated correlation presented in this chapter
reflects the historically dominant put-skew regime.

\section{Risk-Neutral Valuation of Crude Oil Options}

\subsection{Characteristic Function}

The Heston model admits a semi-analytical characteristic function for $\log S_T$. Under
the numerically stable formulation of \citet{lordkahl2010}, which avoids branch-cut
discontinuities in complex logarithms, the characteristic function is:

\begin{equation}
  \varphi(u) = \exp\!\bigl(C(u,T)\,\theta + D(u,T)\,v_0
               + iu\,(\ln S + rT)\bigr)
  \label{eq:char_func}
\end{equation}

where the auxiliary functions $C$ and $D$ are:
\begin{align}
  \xi      &= \kappa - \rho\sigma i u \\
  d        &= \sqrt{\xi^2 + \sigma^2 u(u+i)} \\
  g        &= \frac{\xi - d}{\xi + d} \\
  C(u,T)   &= \frac{\kappa}{\sigma^2}\!\left[(\xi-d)T
              - 2\ln\!\frac{1 - g\,e^{-dT}}{1-g}\right] \\
  D(u,T)   &= \frac{\xi-d}{\sigma^2}\cdot\frac{1 - e^{-dT}}{1 - g\,e^{-dT}}
\end{align}

The Python implementation follows this formulation directly, using NumPy's complex
arithmetic to evaluate the characteristic function over the full frequency grid in a
single vectorised operation:

\begin{lstlisting}[caption={Heston characteristic function with Lord \& Kahl (2010)
                   numerical stability rotation (\texttt{heston\_model.py})},
                   label={lst:char_func}]
def heston_char_func(u, S, T, r, kappa, theta, sigma, rho, v0):
    i    = 1j
    xi   = kappa - rho * sigma * i * u
    d    = np.sqrt(xi**2 + sigma**2 * u * (u + i))
    g    = (xi - d) / (xi + d)
    exp_dT = np.exp(-d * T)

    # Avoid division by zero when g * exp_dT -> 1
    denom = 1.0 - g * exp_dT
    denom = np.where(np.abs(denom) < 1e-14, 1e-14, denom)

    C = kappa * ((xi - d) * T
        - 2.0 * np.log(denom / (1.0 - g))) / sigma**2
    D = ((xi - d) / sigma**2) * (1.0 - exp_dT) / denom

    return np.exp(C * theta + D * v0 + i * u * (np.log(S) + r * T))
\end{lstlisting}

\subsection{Black-76 Formula}

Since crude oil options are written on futures (not spot prices), we use the
\citet{black1976} model as our baseline single-volatility pricer. The Black-76 call
price is:

\begin{equation}
  C^{B76}(F,K,T,r,\sigma) = e^{-rT}\bigl[F\,\mathcal{N}(d_1) - K\,\mathcal{N}(d_2)\bigr]
  \label{eq:black76}
\end{equation}

with:
\begin{equation}
  d_1 = \frac{\ln(F/K) + \tfrac{1}{2}\sigma^2 T}{\sigma\sqrt{T}}, \qquad
  d_2 = d_1 - \sigma\sqrt{T}
\end{equation}

Implied volatilities are recovered from observed prices by numerically inverting
equation~\eqref{eq:black76} using Brent's root-finding method:

\begin{lstlisting}[caption={Black-76 implied volatility inversion via Brent's method
                   (\texttt{heston\_model.py})},
                   label={lst:black76_iv}]
def black76_implied_vol(price, F, K, T, r, option_type="C"):
    if option_type.upper() == "C":
        pricer = lambda sig: black76_call(F, K, T, r, sig) - price
        intrinsic = max(F - K, 0.0) * np.exp(-r * T)
    else:
        pricer = lambda sig: black76_put(F, K, T, r, sig) - price
        intrinsic = max(K - F, 0.0) * np.exp(-r * T)

    if price <= intrinsic + 1e-8:
        return np.nan  # price at or below intrinsic: no solution

    try:
        return brentq(pricer, 1e-6, 5.0, xtol=1e-8, maxiter=200)
    except (ValueError, RuntimeError):
        return np.nan
\end{lstlisting}

\subsection{Carr-Madan FFT Pricing}

To price options at a grid of strikes simultaneously — which is essential for
calibration efficiency — we use the fast Fourier transform approach of
\citet{carrmadan1999}. The damped Fourier transform of the call price is:

\begin{equation}
  \psi(v) = \frac{e^{-rT}\,\varphi\!\bigl(v - (\alpha+1)i\bigr)}
                 {\alpha^2 + \alpha - v^2 + i(2\alpha+1)v}
  \label{eq:cm_psi}
\end{equation}

where $\alpha > 0$ is a damping parameter (set to $\alpha = 1.5$). A single FFT call
on a grid of $N = 4{,}096$ points recovers call prices across the full strike range.

\begin{lstlisting}[caption={Carr-Madan FFT pricer: prices all strikes in one
                   vectorised FFT call (\texttt{heston\_model.py})},
                   label={lst:fft_pricer}]
def heston_call_price_fft(S, K_array, T, r, kappa, theta, sigma, rho, v0,
                           N=4096, eta=0.25, alpha=1.5):
    v = np.arange(N) * eta
    v[0] = 1e-14    # avoid division by zero at v = 0

    # Evaluate characteristic function on Carr-Madan contour
    u   = v - (alpha + 1) * 1j
    phi = heston_char_func(u, S, T, r, kappa, theta, sigma, rho, v0)

    # Damped call price transform (eq. cm_psi)
    psi = (np.exp(-r * T) * phi /
           (alpha**2 + alpha - v**2 + 1j * (2 * alpha + 1) * v))

    # Simpson's rule weights for integration accuracy
    simpson = (3 + (-1)**np.arange(N)
               - np.where(np.arange(N) == 0, 1, 0)) / 3.0
    x = np.exp(1j * np.arange(N) * np.pi) * psi * simpson * eta

    fft_vals = np.real(np.fft.fft(x))

    # Recover log-strike grid and interpolate to requested strikes
    lambda_    = 2 * np.pi / (N * eta)
    log_strikes = -N * lambda_ / 2 + lambda_ * np.arange(N)
    call_prices = np.exp(-alpha * log_strikes) / np.pi * fft_vals

    prices = np.interp(np.log(K_array), log_strikes, call_prices)
    return np.maximum(prices, np.maximum(S - K_array, 0.0) * np.exp(-r * T))
\end{lstlisting}

\section{Calibration Procedure}

\subsection{Data Selection}
\label{sec:data_selection}

On each calibration date, all options for the target underlying are extracted from
the assembled Bloomberg surface (Section~\ref{sec:final_dataset}). Options whose
moneyness falls outside $[0.70,\;1.30]$ are removed to exclude deep in-the-money
and deep out-of-the-money strikes where liquidity is negligible and surface
interpolation is unreliable. On the calibration date of 19 February 2026 this
yields \textbf{42 observations per underlying} (7 moneyness levels $\times$ 6
maturities, spanning 1-month to 24-month tenors). The risk-free rate applied to
each option is the tenor-matched US Treasury yield interpolated from the live
yield curve (Section~\ref{sec:loading_rfr}).

\subsection{Objective Function}

We calibrate the parameter vector $\Theta = (\kappa, \theta, \sigma, \rho, v_0)$ by
minimising the vega-weighted mean squared error (MSE) between model and market implied
volatilities:

\begin{equation}
  \mathcal{L}(\Theta) = \frac{\sum_{i=1}^{N} w_i \bigl(\sigma_i^{\text{model}}(\Theta)
                         - \sigma_i^{\text{market}}\bigr)^2}{\sum_{i=1}^{N} w_i}
  \label{eq:objective}
\end{equation}

Following \citet{conttankov2004}, we use vega-inspired weights that assign greater
importance to at-the-money options, which are more liquidly traded:

\begin{equation}
  w_i = \exp\!\left(-\frac{(K_i/S - 1)^2}{2 \times 0.1^2}\right)
\end{equation}

\begin{lstlisting}[caption={Vega-weighted calibration objective function
                   (\texttt{heston\_model.py})},
                   label={lst:objective}]
def calibration_objective(params, S, K_array, T_array, r,
                           market_ivols, weights=None):
    model_ivols = heston_implied_vol(params, S, K_array, T_array, r)
    valid = ~np.isnan(model_ivols) & ~np.isnan(market_ivols)

    if valid.sum() < 3:
        return 1e6   # too few valid points: penalise heavily

    diff = model_ivols[valid] - market_ivols[valid]

    if weights is None:
        # Gaussian ATM weights (vega-inspired)
        moneyness    = K_array[valid] / S
        weights_use  = np.exp(-0.5 * ((moneyness - 1.0) / 0.1)**2)
    else:
        weights_use = weights[valid]

    return np.sum(weights_use * diff**2) / np.sum(weights_use)
\end{lstlisting}

\subsection{Two-Stage Optimisation}

The MSE surface \eqref{eq:objective} is non-convex and exhibits multiple local minima.
A single gradient-based call from an arbitrary starting point is therefore unreliable.
We adopt a two-stage procedure:

\begin{enumerate}
  \item \textbf{Global search.} We apply differential evolution with a population of
        15 individuals, 300 maximum iterations, and mutation rate in $[0.5, 1.5]$.
        This provides a robust initial estimate near the global minimum.
  \item \textbf{Local refinement.} Starting from the best solution found in Step 1,
        we run L-BFGS-B with up to 1{,}000 iterations and gradient tolerance
        $10^{-8}$ to obtain a precise local minimum.
\end{enumerate}

\begin{lstlisting}[caption={Two-stage Heston calibration: global differential
                   evolution followed by L-BFGS-B refinement
                   (\texttt{heston\_model.py})},
                   label={lst:calibration}]
def calibrate_heston(S, K_array, T_array, r, market_ivols, weights=None):
    bounds = [
        (0.01, 20.0),     # kappa: mean-reversion speed
        (0.001, 2.0),     # theta: long-run variance
        (0.01, 2.0),      # sigma: vol of vol
        (-0.999, 0.999),  # rho:   spot-variance correlation
        (0.001, 2.0),     # v0:    initial variance
    ]
    obj = lambda p: calibration_objective(
        p, S, K_array, T_array, r, market_ivols, weights
    )

    # Stage 1: global search
    de_result = differential_evolution(
        obj, bounds, maxiter=300, seed=42,
        popsize=15, mutation=(0.5, 1.5), recombination=0.7
    )
    # Stage 2: local refinement
    local_result = minimize(
        obj, de_result.x, method="L-BFGS-B", bounds=bounds,
        options={"maxiter": 1000, "ftol": 1e-12, "gtol": 1e-8}
    )

    kappa, theta, sigma, rho, v0 = local_result.x
    feller_val = 2 * kappa * theta - sigma**2
    return {
        "params": local_result.x,
        "kappa": kappa, "theta": theta, "sigma": sigma,
        "rho": rho, "v0": v0,
        "mse": local_result.fun,
        "rmse_volpts": np.sqrt(local_result.fun) * 100,
        "feller": feller_val >= 0,
        "feller_value": feller_val,
    }
\end{lstlisting}

The calibration is performed separately for CO1 (Brent) and CL1 (WTI) on each
calibration date.

% ============================================================
%  SECTION 3.4 — CALIBRATION RESULTS
% ============================================================
\section{Calibration Results}
\label{sec:calibration_results}

\subsection{Calibrated Parameters}

Table~\ref{tab:calibrated_params} reports the calibrated parameter vector
$\hat{\Theta} = (\hat{\kappa},\hat{\theta},\hat{\sigma},\hat{\rho},\hat{v}_0)$
for Brent and WTI on 19 February 2026, together with the root-mean-squared error
(RMSE) in implied volatility points and the Feller diagnostic.

\begin{table}[H]
  \centering
  \caption{Heston calibration results for Brent (CO1) and WTI (CL1),
           19 February 2026. RMSE expressed in implied volatility percentage points.
           Feller value $= 2\hat{\kappa}\hat{\theta} - \hat{\sigma}^2$.}
  \label{tab:calibrated_params}
  \begin{tabular}{lrrl}
    \toprule
    Parameter & Brent (CO1) & WTI (CL1) & Interpretation \\
    \midrule
    $\hat{\kappa}$ & 4.775  & 2.384  & Mean-reversion speed of variance \\
    $\hat{\theta}$ & 0.0363 & 0.0035 & Long-run variance ($\sqrt{\hat{\theta}}$: 19.1\%, 5.9\%) \\
    $\hat{\sigma}$ & 1.589  & 0.763  & Volatility of volatility \\
    $\hat{\rho}$   & $+$0.770 & $+$0.809 & Spot-variance correlation \\
    $\hat{v}_0$    & 0.2143 & 0.1569 & Initial variance ($\sqrt{\hat{v}_0}$: 46.3\%, 39.6\%) \\
    \midrule
    RMSE (vol pts)                          & 1.68\% & 1.61\% & Calibration fit \\
    Feller $2\hat{\kappa}\hat{\theta}-\hat{\sigma}^2$ & $-2.17$ & $-0.55$ & \textbf{Violated} \\
    \bottomrule
  \end{tabular}
\end{table}

\subsection{Smile Cross-Sections}
\label{sec:smile_fit}

Figures~\ref{fig:co1_smile} and~\ref{fig:cl1_smile} show the Heston model smile
against market implied volatilities across all six maturities for Brent and WTI
respectively. Market observations (blue dots) are overlaid with the model curve
(red line) obtained by FFT pricing followed by Black-76 inversion at~$\hat{\Theta}$.

\begin{figure}[H]
  \centering
  \includegraphics[width=\textwidth]{CO1_vol_smile.png}
  \caption{Heston model implied volatility (red line) versus market implied
           volatility (blue dots) for Brent (CO1) across six maturities,
           19 February 2026. Overall RMSE = 1.68 vol pts.
           Source: Bloomberg OVDV; \texttt{calibrate.py}.}
  \label{fig:co1_smile}
\end{figure}

\begin{figure}[H]
  \centering
  \includegraphics[width=\textwidth]{CL1_vol_smile.png}
  \caption{Heston model implied volatility (red line) versus market implied
           volatility (blue dots) for WTI (CL1) across six maturities,
           19 February 2026. Overall RMSE = 1.61 vol pts.
           Source: Bloomberg OVDV; \texttt{calibrate.py}.}
  \label{fig:cl1_smile}
\end{figure}

The cross-sections reveal a systematic pattern: the model fits the ATM region
well at all maturities but underestimates the right wing (OTM calls) and
overestimates the left wing (OTM puts) at short maturities. This arises because
the Heston model generates approximately quadratic smiles in log-moneyness, whereas
the crude oil market exhibits a near-monotonic, positively-skewed smile driven by
the large positive~$\hat{\rho}$. A single parameter vector cannot simultaneously
match the skew level, the term structure of skew, and the smile curvature across
all six tenors — a structural limitation that becomes more pronounced at the 1-month
and 2-month horizons where near-term event risk is concentrated.

\subsection{Volatility Surface Fit}
\label{sec:surface_fit}

Figures~\ref{fig:co1_surface} and~\ref{fig:cl1_surface} present the full implied
volatility surface generated by the calibrated Heston model, overlaid with the
42 market data points for each underlying.

\begin{figure}[H]
  \centering
  \includegraphics[width=\textwidth]{CO1_vol_surface.png}
  \caption{Calibrated Heston implied volatility surface for Brent (CO1),
           19 February 2026. Coloured mesh = model surface generated by the
           Carr-Madan FFT pricer at~$\hat{\Theta}$; red dots = market implied
           volatilities. Source: \texttt{calibrate.py}.}
  \label{fig:co1_surface}
\end{figure}

\begin{figure}[H]
  \centering
  \includegraphics[width=\textwidth]{CL1_vol_surface.png}
  \caption{Calibrated Heston implied volatility surface for WTI (CL1),
           19 February 2026. Source: \texttt{calibrate.py}.}
  \label{fig:cl1_surface}
\end{figure}

\subsection{Economic Interpretation and Model Limitations}
\label{sec:limitations}

\paragraph{Positive correlation.}
The strongly positive $\hat{\rho}$ ($+0.770$ for Brent, $+0.809$ for WTI)
confirms the inverse leverage effect discussed in Section~\ref{sec:heston_oil}.
WTI's higher value is consistent with its greater sensitivity to domestic US
supply shocks, which generate more explosive joint price-variance co-movements
than the globally traded Brent benchmark. Both values lie well into positive
territory, ruling out any equity-like leverage interpretation of the crude oil
options market.

\paragraph{High volatility of volatility.}
The large $\hat{\sigma}$ values (1.589 for Brent, 0.763 for WTI) reflect the
fat-tailed, regime-switching character of crude oil variance documented in
Figure~\ref{fig:atm_history}. High $\hat{\sigma}$ is required for the model to
span the wide level difference between the 1-month and 24-month smiles: the short
end is dominated by near-term event risk, while the long end is governed by
slower-moving structural supply-demand dynamics.

\paragraph{Feller condition violated.}
Both calibrations violate the Feller condition \eqref{eq:feller}:
$2\hat{\kappa}\hat{\theta} - \hat{\sigma}^2 = -2.17$ (CO1) and $-0.55$ (CL1).
This implies that the variance process $V_t$ can reach zero in finite time with
positive probability under the calibrated parameters. Feller violations are
routinely observed when calibrating to volatile, strongly-skewed option surfaces
\citep{broadie2006}. The model remains numerically well-posed in practice because
the FFT pricer floors option prices at intrinsic value; we report the diagnostic
transparently rather than imposing a binding Feller constraint that would
artificially prevent the optimiser from reaching the global minimum.

\paragraph{RMSE and model misspecification.}
An RMSE of $1.6$--$1.7$ implied volatility percentage points is non-trivial in a
market where bid-ask spreads on liquid near-dated contracts are of order
$0.5$--$1.0$ vol pts. The residual fit error reflects genuine model
misspecification: Heston's single continuous Brownian variance driver cannot
simultaneously reproduce the time-varying skew direction — positive under
supply-shock regimes, negative during demand crises (Figure~\ref{fig:risk_reversal})
— and the non-quadratic smile curvature visible at short maturities. This
systematic misfit is not a calibration artefact but an inherent boundary of the
model's parametric family. It is precisely this tension between a five-parameter
model and a structurally complex, regime-dependent surface that motivates the
Hessian spectral analysis in Chapter~\ref{ch:hessian}: by examining the curvature
of $\mathcal{L}(\Theta)$ at $\hat{\Theta}$, we can rigorously quantify which
parameter combinations are well-constrained by the market data and which remain
poorly identified — providing a formal measure of the model's identifiability
limits in crude oil options markets.

% ============================================================
%  CHAPTER 4 — RESULTS AND CONCLUSION
% ============================================================
\chapter{Discussion and Conclusion}
\label{ch:discussion}

\section{Summary of Findings}
\label{sec:summary}

Chapters 1--3 of this dissertation have established the data infrastructure,
the pricing model, and the calibration pipeline for a systematic study of the
Heston stochastic volatility model in crude oil options markets.

The Bloomberg OVDV data (Chapter~\ref{ch:data}) provides daily implied
volatility surfaces for Brent (CO1) and WTI (CL1) over the period January 2006 to
February 2026, covering six constant-maturity tenors and seven moneyness levels. The
empirical analysis in Chapter~\ref{ch:loading} reveals three robust stylised
facts: a predominantly negative (put-skew) implied volatility skew that inverted to
positive after mid-2022 (Figure~\ref{fig:avg_smile_shape}), sharp smile
distortions during market stress episodes (Figure~\ref{fig:crisis_smile}), and a
time-varying risk reversal that periodically changes sign between supply-driven and
demand-driven crises (Figure~\ref{fig:risk_reversal}).

The Heston (1993) model was calibrated to a static snapshot of 1 December 2021 ---
the most recent date within the historically dominant put-skew regime ---
using the Carr-Madan FFT pricer and a two-stage global-plus-local optimisation
procedure (Chapter~\ref{ch:methodology}). The calibrated parameters are reported in
Table~\ref{tab:calibrated_params} and the goodness-of-fit in
Figures~\ref{fig:co1_smile}--\ref{fig:cl1_surface}. In brief: both underlyings exhibit
negative spot-variance correlation (consistent with put-skew dominance), large vol-of-vol,
Feller condition violations, and a residual RMSE of approximately 1.5--2.7 implied
volatility percentage points.

\section{Comparative Analysis: Brent vs WTI}
\label{sec:brent_vs_wti}

Despite trading the same underlying commodity --- crude oil --- the Brent and WTI markets
produce meaningfully different calibrated parameter vectors, reflecting their distinct
microstructures and geopolitical exposures.

\paragraph{Mean-reversion speed.}
Brent's $\hat{\kappa} = 0.335$ and WTI's $\hat{\kappa} = 0.045$ are both low on
1 December 2021, consistent with a relatively calm, post-COVID recovery environment
where transient volatility shocks dissipate slowly and the variance process hugs its
long-run level. The large difference in $\hat{\kappa}$ between the two contracts
indicates that Heston's mean-reversion parameterisation is sensitive to the specific
surface shape on any given date, confirming the chronic sloppy-$\kappa$ finding
established in Chapter~\ref{ch:hessian}.

\paragraph{Long-run variance.}
Both $\hat{\theta}$ values are close to their calibration lower bound ($0.001$),
reflecting the low volatility regime of late 2021. This is a known artefact of
Heston calibration: when the variance process is calibrated during a low-vol period,
$\hat{\theta}$ and $\hat{\kappa}$ compensate jointly for the short-end of the
term structure, making their individual estimates unreliable.

\paragraph{Volatility of volatility.}
Brent's $\hat{\sigma} = 1.109$ and WTI's $\hat{\sigma} = 2.0$ (at the calibration
upper bound for WTI). A high $\hat{\sigma}$ generates the steep left wing of the put
skew, consistent with the negative $\hat{\rho}$ and the demand-driven left-tail risk
of crude oil at this date.

\paragraph{Spot-variance correlation.}
Both contracts produce a large negative $\hat{\rho}$: $-0.796$ (Brent) and $-0.773$
(WTI). This confirms the historically dominant put-skew regime: demand-crash tail risk
outweighs supply-spike upside risk on 1 December 2021. The two markets agree closely
on the direction of $\hat{\rho}$ despite differing in all other parameters, providing
independent confirmation of the put-skew structure documented in
Chapter~\ref{ch:loading}. Note that this structure has since reversed: since mid-2022,
rolling calibrations consistently produce $\hat{\rho} > 0$, signalling a persistent
shift to call-skew dominance as discussed in Section~\ref{sec:heston_oil}.

\paragraph{Initial variance.}
$\hat{v}_0 = 0.357$ for Brent ($59.7\%$ initial vol) and $0.432$ for WTI ($65.7\%$),
both substantially above the near-zero long-run estimates. The elevated initial
variance reflects the realised volatility of late 2021 (Omicron wave uncertainty,
energy supply concerns ahead of the Ukraine conflict) and the model's need to
match the short-end of the term structure independently of the long-run $\hat{\theta}$.

\section{Calibration Findings and Market Stylised Facts}
\label{sec:calibration_vs_stylised}

A central question is whether the calibrated Heston parameters are \emph{consistent}
with the empirical regularities documented in Chapter~\ref{ch:loading}. The answer
is yes in direction but limited in magnitude.

\paragraph{Put skew explained.}
The negative $\hat{\rho} \approx -0.80$ directly produces the downward-sloping
(put-skew) smile characteristic of crude oil options during the 2006--2022 period:
a negative correlation between price and variance tilts the risk-neutral distribution
leftward, inflating put implied volatilities relative to equidistant calls. The
time-averaged negative skew shown in Figure~\ref{fig:avg_smile_shape} is consistent
with this calibrated direction. The recent (post-2022) positive-skew regime — visible
in the right tail of Figure~\ref{fig:risk_reversal} — would require $\hat{\rho} > 0$
and is not the focus of the static analysis presented here.

\paragraph{Term structure of skew.}
The attenuation of the skew at longer maturities in Figure~\ref{fig:avg_smile_shape} is
qualitatively reproduced by the Heston model: as $T \to \infty$, the risk-neutral
distribution converges to the long-run distribution governed by $\hat{\theta}$, and the
correlation effect diminishes. However, the Heston model implies a specific functional
form for this attenuation that need not match the market's own decay rate across
maturities.

\paragraph{Crisis distortions not fully reproducible.}
The crisis-period smile distortions in Figure~\ref{fig:crisis_smile} present a
fundamental challenge for any single-date calibration. During COVID-19 (April 2020),
the market smile at 1M became sharply left-skewed — consistent with the negative
$\hat{\rho}$ of the 1 December 2021 calibration but much more extreme than the model
can reproduce. During the post-2022 supply-driven episode, the smile inverts to
right-skewed, requiring $\hat{\rho} > 0$. A fixed-parameter Heston model calibrated to
a single date cannot simultaneously match both regimes. This structural inability is
precisely what makes the Hessian condition number an interesting quantity to study across
time: we expect it to change substantially during regime transitions.

\paragraph{Risk reversal dynamics.}
Figure~\ref{fig:risk_reversal} shows that the 1M risk reversal changes sign multiple
times over the 2006--2026 sample. Each sign change corresponds to a qualitative change
in the smile shape that is irreconcilable within a single Heston parameterisation. Any
rolling calibration would need to flip $\hat{\rho}$ from positive to negative at crisis
peaks, implying extreme parameter instability and poor out-of-sample performance ---
exactly the kind of ill-conditioning that the Hessian spectral analysis is designed to
quantify.

\section{Model Limitations and Misspecification}
\label{sec:model_limits}

The Heston model is a single-factor, continuous-path, affine stochastic volatility
model. Its limitations in the crude oil context are structural rather than
computational.

\paragraph{Smile shape.}
The model generates an approximately symmetric, quadratic smile in log-moneyness. The
crude oil market consistently exhibits an asymmetric, near-monotone positive skew at
short maturities. No parameter combination can make the Heston model generate a
monotonically increasing smile: the shape mismatch is a topological property of the
parametric family. The optimiser compensates by biasing the smile minimum off-grid, but
the resulting fit error (1.6--1.7 vol pts) exceeds the bid-ask spread for liquid
near-dated contracts.

\paragraph{Jump risk.}
Supply disruptions and geopolitical events produce non-Gaussian jumps in both price and
volatility that are absent from the Heston model. Jump-diffusion extensions (Bates 1996;
\citealt{conttankov2004}) typically reduce RMSE by 30--50\% in equity markets, and a
similar gain is plausible for crude oil. However, adding jump parameters substantially
worsens the identifiability problem that is this dissertation's primary focus.

\paragraph{Stochastic long-run mean.}
The divergence between $\hat{\theta}$ for Brent and WTI suggests that the long-run
variance is itself time-varying --- a two-factor model \citep{trolle2009} would likely fit
the 12-month and 24-month tenors significantly better. Again, this comes at the cost of
additional parameters and reduced identifiability.

\paragraph{Single-regime.}
The risk reversal analysis (Section~\ref{sec:calibration_vs_stylised}) shows that the
market cycles between distinct smile regimes. A regime-switching extension of the Heston
model \citep{conttankov2004} would address this, but the number of
calibration-relevant parameters would increase substantially.

\section{Bridge to Spectral Analysis}
\label{sec:hessian_outlook}

The calibration results of Chapter~\ref{ch:methodology} establish the starting point
for the core methodological contribution of this dissertation: the \emph{spectral
analysis of the Hessian of the calibration loss function}, carried out in full in
Chapter~\ref{ch:hessian}.

\paragraph{Motivation.}
The RMSE of $\approx 1.65$ vol pts and the structural smile mismatch documented above
suggest that the Heston parameter vector is not strongly identified by the market data:
there likely exist alternative parameter combinations that produce a similar calibration
error. To quantify this formally, Chapter~\ref{ch:hessian} computes the Hessian of
$\mathcal{L}(\Theta)$ at $\hat{\Theta}$ by central finite differences and decomposes it
spectrally:
\begin{equation}
  H = V \Lambda V^{\top}, \quad
  \Lambda = \mathrm{diag}(\lambda_1 \geq \lambda_2 \geq \cdots \geq \lambda_5)
  \label{eq:hessian_spec}
\end{equation}
where the columns of $V$ are eigenvectors of $H$ and the diagonal entries of $\Lambda$
are eigenvalues. Large eigenvalues correspond to \emph{stiff} directions in parameter
space --- combinations of parameters to which the loss is highly sensitive and that are
therefore well-constrained by the data. Small eigenvalues correspond to \emph{sloppy}
directions --- parameter combinations that can be varied substantially without degrading
the fit.

\paragraph{Key results (Chapter~\ref{ch:hessian}).}
The static spectral analysis on 19 February 2026 confirms the anticipated ill-conditioning:
\begin{itemize}
  \item The stiffest direction is dominated by $\hat{\theta}$ and $\hat{v}_0$, which
        jointly control the level and initial value of the variance surface.
  \item The sloppiest direction is almost purely $\hat{\kappa}$ (loading $\approx 0.99$)
        --- the mean-reversion speed, which has negligible discriminating power in the
        available surface data.
  \item Condition numbers of $\kappa_H = 3.19 \times 10^4$ (Brent) and
        $4.99 \times 10^4$ (WTI) confirm extreme calibration ill-conditioning, rising
        to median values of $\approx 10^{5.6}$ in the monthly rolling analysis.
\end{itemize}

\paragraph{Rolling analysis (Chapter~\ref{ch:hessian}).}
Rolling the Hessian across 241 monthly dates (2006--2026) reveals that identifiability
does not deteriorate uniformly during crises. Instead, the relationship between market
stress and condition number is negative: higher volatility surfaces carry more geometric
information, reducing $\kappa_H$. This counterintuitive result is quantified formally
in Chapter~\ref{ch:macro} via regression on OVX, VIX, GPR, inventories, and DXY.

\paragraph{Broader implications.}
A high condition number implies that small perturbations in the input implied volatility
data (due to bid-ask noise, stale quotes, or data-cleaning artefacts) can produce large
changes in the calibrated parameters. This has direct practical consequences for risk
management and hedging: vega hedges computed from an ill-conditioned Heston calibration
may be unreliable, and dynamic rebalancing based on rolling calibration may amplify
rather than reduce parameter estimation error. The spectral analysis provides a
principled, geometry-based framework for diagnosing and reporting this risk.

\section{Conclusion}
\label{sec:conclusion}

This dissertation has calibrated the Heston (1993) stochastic volatility model to
twenty years of Bloomberg OVDV implied volatility data for Brent and WTI crude oil
futures options. The data pipeline (\texttt{surface\_loader.py}) parses real Bloomberg
OVDV surfaces, futures settlement prices, and US Treasury yields into a
calibration-ready dataset of 407{,}679 observations. The calibration pipeline
(\texttt{calibrate.py}) combines the Carr-Madan FFT pricer with a two-stage
global-plus-local optimiser to recover the Heston parameter vector for each underlying
on the snapshot date of 19 February 2026.

The principal empirical findings are:
\begin{itemize}
  \item Crude oil options exhibit a persistent \emph{positive} spot-variance correlation
        ($\hat{\rho} \approx +0.77$--$+0.81$), reflecting the inverse leverage effect of
        supply-shock-driven price-variance co-movements.
  \item The Heston model achieves an RMSE of 1.61--1.68 implied volatility percentage
        points --- non-trivial relative to market bid-ask spreads --- due to a structural
        mismatch between the model's quadratic smile and the market's monotone positive
        skew.
  \item The Feller condition is violated for both underlyings, a common consequence of
        calibrating to strongly-skewed option surfaces.
  \item The risk reversal time series (Figure~\ref{fig:risk_reversal}) shows repeated
        sign reversals between supply- and demand-driven crises, demonstrating that a
        single-regime Heston model cannot track the full history of the crude oil
        volatility surface.
\end{itemize}

These findings motivate --- and are complemented by --- the subsequent chapters of this
dissertation. Chapter~\ref{ch:hessian} conducts a Hessian-based spectral analysis of the
calibration landscape, quantifying the identifiability of each parameter combination and
tracking how it evolves across twenty years of market regimes. Chapter~\ref{ch:macro}
then links the rolling condition number to five macro-financial indicators --- OVX, VIX,
geopolitical risk, EIA inventories, and the US dollar --- providing an empirical
characterisation of the market conditions under which the Heston model is well- or
poorly-identified. Together, these analyses constitute a principled, mathematically
rigorous assessment of the Heston model's identifiability limits in crude oil markets.

% ============================================================
\chapter{Hessian Spectral Geometry of the Calibration Landscape}
\label{ch:hessian}

The preceding chapters established the data infrastructure and calibrated the Heston
model to Bloomberg OVDV surfaces for Brent (CO1) and WTI (CL1). Chapter~\ref{ch:discussion}
concluded that the model's structural mismatch with the crude oil surface — a residual
RMSE of $\approx 1.65$ implied volatility percentage points and a Feller condition
violation for both underlyings — suggests significant parameter ill-conditioning.
This chapter quantifies that ill-conditioning directly through the spectral geometry
of the calibration loss function.

\section{Numerical Hessian via Finite Differences}
\label{sec:hessian_method}

\subsection{Central Difference Approximation}
\label{sec:central_diff}

Let $\mathcal{L}(\Theta)$ denote the vega-weighted mean-squared error between Heston
model and market implied volatilities, evaluated at the parameter vector
$\Theta = (\kappa,\,\theta,\,\sigma,\,\rho,\,v_0)^\top \in \mathbb{R}^5$. At the
calibrated optimum $\hat{\Theta}$, the second-order Taylor expansion is:
\begin{equation}
  \mathcal{L}(\hat{\Theta} + \delta) \approx \mathcal{L}(\hat{\Theta})
  + \nabla\mathcal{L}(\hat{\Theta})^\top \delta
  + \tfrac{1}{2}\,\delta^\top H\,\delta,
  \label{eq:taylor_expansion}
\end{equation}
where $H = \nabla^2 \mathcal{L}(\hat{\Theta}) \in \mathbb{R}^{5 \times 5}$ is the
Hessian. Since $\hat{\Theta}$ is a local minimum, $\nabla\mathcal{L}(\hat{\Theta})
\approx \mathbf{0}$ and equation~\eqref{eq:taylor_expansion} shows that the Hessian
fully characterises the local loss landscape.

We compute $H$ by central finite differences. Diagonal elements:
\begin{equation}
  H_{ii} = \frac{\mathcal{L}(\hat{\Theta} + \varepsilon_i \mathbf{e}_i)
                 - 2\,\mathcal{L}(\hat{\Theta})
                 + \mathcal{L}(\hat{\Theta} - \varepsilon_i \mathbf{e}_i)}
                {\varepsilon_i^2},
  \label{eq:hessian_fd}
\end{equation}
where $\mathbf{e}_i$ is the $i$-th unit vector. Off-diagonal elements:
\begin{equation}
  H_{ij} = \frac{\mathcal{L}(\hat{\Theta}{+}\varepsilon_i\mathbf{e}_i{+}\varepsilon_j\mathbf{e}_j)
               - \mathcal{L}(\hat{\Theta}{+}\varepsilon_i\mathbf{e}_i{-}\varepsilon_j\mathbf{e}_j)
               - \mathcal{L}(\hat{\Theta}{-}\varepsilon_i\mathbf{e}_i{+}\varepsilon_j\mathbf{e}_j)
               + \mathcal{L}(\hat{\Theta}{-}\varepsilon_i\mathbf{e}_i{-}\varepsilon_j\mathbf{e}_j)}
                {4\,\varepsilon_i\,\varepsilon_j},
  \label{eq:hessian_fd_offdiag}
\end{equation}
for $i \neq j$. The total cost is $1 + 2\times 5 + 4\binom{5}{2} = 51$ evaluations of
$\mathcal{L}$, each requiring one Carr-Madan FFT pass per maturity (6 FFT calls).

\subsection{Adaptive Step Selection}
\label{sec:adaptive_step}

The finite difference step $\varepsilon_i$ must balance truncation error (too large)
against floating-point cancellation (too small). We use an adaptive rule:
\begin{equation}
  \varepsilon_i = \max\!\bigl(10^{-5},\; |\hat{\theta}_i| \times 10^{-4}\bigr),
  \label{eq:adaptive_step}
\end{equation}
which scales the step to the magnitude of each parameter. This prevents the very small
parameters ($\hat{\theta} \approx 0.004$ for WTI) from driving $\varepsilon$ below
machine-precision thresholds.

\subsection{Implementation}
\label{sec:hessian_code}

The Hessian computation is implemented in \texttt{hessian.py}. The core function
\texttt{compute\_hessian()} calls \texttt{heston\_model.calibration\_objective()} at
each perturbed parameter vector:

\begin{lstlisting}[caption={Core finite-difference Hessian (\texttt{hessian.py})},
                   label={lst:hessian_core}]
def compute_hessian(params, S, K_array, T_array, r, market_ivols,
                    weights=None, rel_eps=1e-4):
    eps = np.maximum(1e-5, np.abs(params) * rel_eps)
    L0  = calibration_objective(params, ...)

    for i in range(5):                      # diagonal
        H[i,i] = (L(+ei) - 2*L0 + L(-ei)) / eps[i]**2

    for i,j pairs:                          # off-diagonal
        H[i,j] = (L(+ei+ej) - L(+ei-ej)
                - L(-ei+ej) + L(-ei-ej)) / (4*eps[i]*eps[j])
    return H
\end{lstlisting}

The spectral decomposition uses \texttt{numpy.linalg.eigh}, which exploits symmetry
and guarantees real eigenvalues:

\begin{lstlisting}[caption={Spectral decomposition (\texttt{hessian.py})},
                   label={lst:spectral}]
eigenvalues, eigenvectors = np.linalg.eigh(H)
idx = np.argsort(eigenvalues)[::-1]   # descending: stiffest first
condition_number = eigenvalues[0] / max(abs(eigenvalues[-1]), 1e-12)
\end{lstlisting}

The condition number is defined as:
\begin{equation}
  \kappa_H = \frac{\lambda_{\max}}{|\lambda_{\min}|}.
  \label{eq:condition_number}
\end{equation}
A large $\kappa_H$ indicates that the loss function has very different curvatures
in different directions of parameter space: some combinations of parameters are tightly
constrained (high curvature, stiff), while others can vary widely without meaningfully
degrading the fit (low curvature, sloppy) \citep{transtrum2010}.

\section{Static Spectral Analysis --- 19 February 2026}
\label{sec:static_spectral}

\subsection{Hessian Matrix Structure}
\label{sec:hessian_matrix}

Figure~\ref{fig:hessian_matrix} displays the Hessian matrices for Brent (CO1, left)
and WTI (CL1, right), evaluated at their respective calibrated parameter vectors on
19 February 2026. Both matrices share the same qualitative structure: the dominant
entries are concentrated in the $(\theta, v_0)$ block, with the $\kappa$ row and column
being two to three orders of magnitude smaller than the rest.

\begin{figure}[H]
  \centering
  \includegraphics[width=\textwidth]{hessian_matrix.png}
  \caption{Hessian of the calibration loss $\mathcal{L}(\Theta)$ at $\hat{\Theta}$,
           19 February 2026. Colour scale is symmetric around zero; red = positive
           (convex), blue = negative (saddle curvature). The $(\theta, v_0)$ block
           dominates in both underlyings.}
  \label{fig:hessian_matrix}
\end{figure}

The large positive diagonal entries $H_{\theta\theta}$ and $H_{v_0 v_0}$ indicate that
the loss is highly sensitive to perturbations in the long-run variance and the initial
variance: even small deviations from $\hat{\theta}$ or $\hat{v}_0$ rapidly degrade the
fit. By contrast, $H_{\kappa\kappa}$ is two to three orders of magnitude smaller,
confirming that the mean-reversion speed is poorly pinned by the cross-sectional
implied volatility data. The off-diagonal entry $H_{\theta,v_0}$ is large and positive,
reflecting the strong co-movement between long-run and initial variance in the
loss landscape: these two parameters partially substitute for one another in fitting
the term structure of ATM implied volatility.

\subsection{Loss Landscape Geometry}
\label{sec:loss_landscape}

Before turning to the full spectral decomposition, it is instructive to visualise the
calibration loss surface directly by evaluating the RMSE across a two-dimensional
grid while fixing the remaining three parameters at their optimal values
$\hat{\Theta}$.  Figure~\ref{fig:loss_landscape} presents four such slices for
Brent (CO1) on 19 February 2026; the $z$-axis is $\log_{10}(\text{RMSE in vol pts})$
to compress the dynamic range.

\begin{figure}[H]
  \centering
  \begin{subfigure}[t]{0.49\textwidth}
    \includegraphics[width=\textwidth]{loss_kappa_sigma.png}
    \caption{$\kappa$--$\xi$ slice. The valley is elongated along the $\kappa$
             axis, confirming that a wide range of mean-reversion speeds yields
             near-identical loss.}
    \label{fig:loss_kappa_sigma}
  \end{subfigure}
  \hfill
  \begin{subfigure}[t]{0.49\textwidth}
    \includegraphics[width=\textwidth]{loss_rho_sigma.png}
    \caption{$\rho$--$\xi$ slice. The surface shows a curved ridge: $\rho$ and
             $\xi$ jointly govern skew, creating a partially compensating
             trade-off.}
    \label{fig:loss_rho_sigma}
  \end{subfigure}
  \vspace{0.5em}
  \begin{subfigure}[t]{0.49\textwidth}
    \includegraphics[width=\textwidth]{loss_kappa_theta.png}
    \caption{$\kappa$--$\theta$ slice. The loss is steep in $\theta$ (long-run
             variance is tightly identified) but flat along $\kappa$.}
    \label{fig:loss_kappa_theta}
  \end{subfigure}
  \hfill
  \begin{subfigure}[t]{0.49\textwidth}
    \includegraphics[width=\textwidth]{loss_rho_v0.png}
    \caption{$\rho$--$v_0$ slice. Both parameters are comparatively well
             identified; the minimum is compact and the red dot (optimal
             $\hat{\Theta}$) sits at the base of a clear bowl.}
    \label{fig:loss_rho_v0}
  \end{subfigure}
  \caption{2D slices of the Heston calibration loss landscape ($\log_{10}$\,RMSE,
           vol pts) for Brent CO1 on 19 February 2026. In each panel the
           remaining three parameters are fixed at $\hat{\Theta}$; the red dot
           marks the calibrated optimum. The pronounced flatness along $\kappa$
           in panels (a) and (c) is the geometric signature of the sloppy
           direction identified by the Hessian spectral analysis.}
  \label{fig:loss_landscape}
\end{figure}

Several features of Figure~\ref{fig:loss_landscape} deserve emphasis.  First, the
$\kappa$--$\xi$ and $\kappa$--$\theta$ slices (panels a and c) both display an
extended, nearly flat valley along the $\kappa$ axis: RMSE barely changes as
$\kappa$ moves from 1 to 10, confirming visually the near-zero Hessian curvature
in that direction.  Second, the $\rho$--$\xi$ slice (panel b) exhibits a curved
ridge rather than a bowl, indicating a compensating relationship between correlation
and vol-of-vol — both govern skew, and the surface cannot fully separate their
individual contributions from the observable strike–maturity grid.  Third, the
$\rho$--$v_0$ slice (panel d) is the most regular: a compact bowl with a clear
global minimum, consistent with the high eigenvalues associated with these two
parameters in the spectral decomposition below.  Together, the four slices provide
a direct, intuitive account of why condition numbers of order $10^4$--$10^5$ emerge
from a model that, in principle, has only five parameters.

\subsection{Eigenvalue Spectrum}
\label{sec:eigenvalue_spectrum}

Table~\ref{tab:hessian_spectrum} and Figure~\ref{fig:eigenvalue_spectrum} report the
complete eigenvalue spectrum for both underlyings. The stiff-sloppy structure is
dramatic: the ratio of the largest to smallest eigenvalue exceeds $3 \times 10^4$ for
Brent and $5 \times 10^4$ for WTI.

\begin{table}[H]
  \centering
  \caption{Eigenvalue spectrum of the Hessian and condition number at $\hat{\Theta}$,
           19 February 2026. Dominant parameter: the parameter with the largest
           absolute loading in the corresponding eigenvector.}
  \label{tab:hessian_spectrum}
  \begin{tabular}{lrrrrr}
    \toprule
    & \multicolumn{2}{c}{CO1 (Brent)} & & \multicolumn{2}{c}{CL1 (WTI)} \\
    \cmidrule(lr){2-3} \cmidrule(lr){5-6}
    Mode & Eigenvalue & Dominant & & Eigenvalue & Dominant \\
    \midrule
    $\lambda_1$ (stiffest) & $2.985$ & $\theta$   & & $4.845$        & $\theta$  \\
    $\lambda_2$            & $0.601$ & $v_0$      & & $1.083$        & $v_0$     \\
    $\lambda_3$            & $0.014$ & $\rho$     & & $5.84\times10^{-3}$ & $\sigma$ \\
    $\lambda_4$            & $2.66\times10^{-3}$ & $\sigma$ & & $8.54\times10^{-4}$ & $\rho$ \\
    $\lambda_5$ (sloppiest)& $9.36\times10^{-5}$ & $\kappa$ & & $9.71\times10^{-5}$ & $\kappa$ \\
    \midrule
    $\kappa_H = \lambda_1 / |\lambda_5|$ & $\mathbf{3.19 \times 10^4}$ & & &
                                           $\mathbf{4.99 \times 10^4}$ & \\
    \bottomrule
  \end{tabular}
\end{table}

\begin{figure}[H]
  \centering
  \includegraphics[width=\textwidth]{eigenvalue_spectrum.png}
  \caption{Eigenvalue spectrum of the Hessian (log scale), 19 February 2026. Each bar
           is labelled with the dominant parameter loading. The five-decade gap between
           $\lambda_1$ and $\lambda_5$ quantifies the stiff-sloppy hierarchy.}
  \label{fig:eigenvalue_spectrum}
\end{figure}

The five eigenvalues span more than four orders of magnitude for both underlyings, a
hallmark of sloppy models in the sense of \citet{transtrum2010}. The loss landscape
resembles a narrow canyon: movement along the stiff ($\theta$, $v_0$) directions is
immediately penalised, while movement along the sloppy ($\kappa$) direction changes
$\mathcal{L}$ by less than $10^{-4}$ of the change induced by the stiffest direction.

\subsection{Stiff and Sloppy Parameter Directions}
\label{sec:eigenvectors}

Figure~\ref{fig:eigenvector_composition} decomposes each eigenvector into its five
parameter loadings, revealing which combinations of parameters are stiff (well
identified) and which are sloppy (poorly identified).

\begin{figure}[H]
  \centering
  \includegraphics[width=\textwidth]{eigenvector_composition.png}
  \caption{Signed eigenvector loadings for each eigenmode, 19 February 2026.
           The stiffest mode ($\lambda_1$) is a mixture of $\theta$ and $v_0$;
           the sloppiest mode ($\lambda_5$) is almost purely $\kappa$.
           Colours: $\kappa$ (blue), $\theta$ (green), $\sigma$ (orange),
           $\rho$ (red), $v_0$ (purple).}
  \label{fig:eigenvector_composition}
\end{figure}

Three structural findings emerge from Figure~\ref{fig:eigenvector_composition}:

\paragraph{Stiffest direction: long-run and initial variance.}
The stiffest eigenmode ($\lambda_1 \approx 3$--$5$) has eigenvector loadings of
approximately $+0.89$ on $\theta$ and $+0.45$ on $v_0$ for Brent, and similarly for
WTI. This direction is the ``variance level'' combination: raising both the initial and
long-run variance simultaneously produces the largest increase in $\mathcal{L}$. The
market data (six maturities × seven moneyness levels per snapshot) provides strong
information about the level of implied volatility, making this direction well
constrained.

\paragraph{Second mode: variance term structure.}
The second eigenmode ($\lambda_2 \approx 0.6$--$1.1$) loads primarily on $v_0$ with a
sign opposite to $\theta$: approximately $-0.45$ on $\theta$ and $+0.89$ on $v_0$ for
Brent. This direction captures the \emph{slope} of the term structure: a position
with high $v_0$ and low $\theta$ (or vice versa) represents a steeply declining (or
rising) term structure. This direction is also well constrained, with 24-month options
in the dataset anchoring the long end.

\paragraph{Sloppiest direction: mean-reversion speed.}
The sloppiest eigenmode ($\lambda_5 \approx 10^{-4}$) is almost entirely $\kappa$ (loading $\approx 0.99$)
for both underlyings. The mean-reversion speed affects how quickly $V_t$ returns from
$v_0$ toward $\theta$, but from a single cross-sectional snapshot, $\kappa$ is largely
degenerate with the pair $(\theta, v_0)$: different values of $\kappa$ can be
approximately offset by adjusting the initial and long-run variances. This fundamental
degeneracy explains why the calibration is highly unstable with respect to $\kappa$
across dates.

\subsection{Condition Number and Calibration Stability}
\label{sec:condition_static}

The condition numbers $\kappa_H = 3.19 \times 10^4$ (Brent) and
$4.99 \times 10^4$ (WTI) quantify the ill-conditioning in a single scalar. A perturbation
$\delta\mathcal{L}$ in the loss value (due to, e.g., bid-ask noise in the input
implied volatilities) can cause a parameter perturbation as large as:
\begin{equation}
  \|\delta\hat{\Theta}\|_2 \;\leq\; \kappa_H \cdot \frac{\|\delta\mathcal{L}\|}{\|\mathcal{L}\|}.
\end{equation}
With $\kappa_H \approx 5 \times 10^4$, even a $0.01\%$ change in the loss level can
in principle produce a $500\%$ change in the calibrated parameter vector --- making
the calibration highly sensitive to input data quality and numerical precision.

\section{Rolling Hessian Analysis --- 2006 to 2026}
\label{sec:rolling_hessian}

\subsection{Rolling Calibration Methodology}
\label{sec:rolling_method}

To track how the calibration landscape evolves over time, we compute the Hessian and
its spectral decomposition at monthly snapshots from January 2006 to February 2026.
Each snapshot corresponds to the first available trading day of each calendar month,
yielding approximately 240 dates per underlying (480 total calibrations).

\paragraph{Warm-start strategy.}
Full global calibration via differential evolution at every date would be computationally
prohibitive. Instead, we use a warm-start approach: the first date in the sample uses
the full two-stage procedure (\texttt{calibrate\_heston}); subsequent dates use
L-BFGS-B only, initialised from the previous month's calibrated parameters. If the
warm-start RMSE exceeds 5 implied volatility percentage points (indicating a local
minimum), the algorithm falls back to full differential evolution. This strategy is
standard in rolling calibration workflows and reduces computation time by an order
of magnitude while preserving accuracy during stable market regimes.

\paragraph{Checkpointing.}
Results are appended to \texttt{rolling\_hessian.csv} every 10 dates, enabling restart
after interruption without repeating completed computations.

\subsection{Condition Number Through Market Regimes}
\label{sec:condition_rolling}

Figure~\ref{fig:condition_number_ts} displays the rolling condition number
$\kappa_H(t)$ for Brent (solid) and WTI (dashed) from 2006 to 2026. This is the
dissertation's headline empirical result.

\begin{figure}[H]
  \centering
  \includegraphics[width=\textwidth]{condition_number_ts.png}
  \caption{Rolling condition number $\kappa_H = \lambda_{\max} / |\lambda_{\min}|$ of
           the Hessian, monthly snapshots 2006--2026 (log scale). Crisis episodes
           (GFC 2008, Supply Glut 2016, COVID-19 2020, Ukraine 2022) are marked by
           dashed vertical lines. Brent (CO1) solid seagreen; WTI (CL1) dashed
           steelblue.}
  \label{fig:condition_number_ts}
\end{figure}

The time series reveals three consistent patterns.

\paragraph{Persistently high conditioning.}
The condition number remains above $10^3$ throughout the 2006--2026 sample for both
underlyings. There is no extended period during which the Heston calibration is
well-conditioned: even in the calmest pre-GFC years, $\kappa_H \gtrsim 10^3$,
confirming that the structural mismatch between the Heston smile shape and the crude
oil surface produces persistent ill-conditioning.

\paragraph{Crisis spikes.}
Each major market disruption is accompanied by a pronounced spike in $\kappa_H$. During
the GFC (2008--2009), the COVID-19 crash (March--April 2020), and the Ukraine-driven
supply shock (March 2022), the condition number rises sharply above its surrounding
baseline. This is consistent with the risk reversal analysis of
Section~\ref{sec:calibration_vs_stylised}: during crises, the smile shape changes
qualitatively (flipping from positive to negative skew, or amplifying the normal
positive skew sharply), forcing the calibration into a parameter region where the
Heston model is structurally misspecified.

\paragraph{Brent vs WTI.}
WTI's condition number is consistently higher than Brent's, matching the static finding
($\kappa_H^{\text{WTI}} \approx 4.99 \times 10^4$ vs $3.19 \times 10^4$ at the
2026 snapshot). This is consistent with WTI's greater sensitivity to idiosyncratic
US supply disruptions, which create sharper smile distortions that the Heston model
struggles to accommodate.

\subsection{Eigenvalue Dynamics}
\label{sec:eigenvalue_rolling}

Figure~\ref{fig:eigenvalue_ts} decomposes the condition number variation into its five
constituent eigenvalues over time.

\begin{figure}[H]
  \centering
  \includegraphics[width=\textwidth]{eigenvalue_ts.png}
  \caption{Rolling eigenvalue time series for Brent (top) and WTI (bottom), monthly
           2006--2026 (log scale). The stiffest eigenvalue $\lambda_1$ (dominated by
           $\theta$ and $v_0$) varies most with market conditions; the sloppiest
           eigenvalue $\lambda_5$ (dominated by $\kappa$) remains small and relatively
           stable.}
  \label{fig:eigenvalue_ts}
\end{figure}

The dominant driver of $\kappa_H$ variation is $\lambda_1$: during crisis periods,
the stiffest eigenvalue (long-run and initial variance direction) increases sharply,
reflecting the heightened level sensitivity of the loss as the market moves into a
distinctly different volatility regime. By contrast, $\lambda_5$ (the $\kappa$
direction) is relatively stable through time --- the mean-reversion speed remains
persistently sloppy regardless of market regime.

\subsection{Rotation of the Sloppy Direction}
\label{sec:sloppy_rotation}

Figure~\ref{fig:sloppy_direction_ts} tracks the composition of the sloppiest
eigenvector ($\lambda_5$) over time.

\begin{figure}[H]
  \centering
  \includegraphics[width=\textwidth]{sloppy_direction_ts.png}
  \caption{Composition of the sloppiest eigenvector ($\lambda_5$) over time for Brent
           (top) and WTI (bottom). The stacked area shows the signed loading of each
           parameter on the sloppy direction. $\kappa$ (blue) dominates throughout;
           during crisis periods the loadings of $\rho$ and $\sigma$ increase,
           indicating a partial rotation of the sloppy direction.}
  \label{fig:sloppy_direction_ts}
\end{figure}

The sloppy direction is dominated by $\kappa$ ($\approx 0.99$ loading) in calm
periods. During crisis episodes, particularly COVID-19 and the GFC, the loading of
$\rho$ and $\sigma$ on the sloppy direction increases materially. This indicates that
during severe market dislocations, the spot-variance correlation and the vol-of-vol
also become difficult to identify from the cross-sectional surface --- the model is
struggling to fit multiple aspects of the smile shape simultaneously and must sacrifice
identifiability in the skew-and-curvature parameters as well.

\section{Discussion}
\label{sec:hessian_discussion}

\subsection{Parameter Identifiability}
\label{sec:identifiability}

The spectral results confirm the informal intuition from Chapter~\ref{ch:discussion}
with quantitative precision. The Heston model's five parameters are \emph{not equally
identifiable} from a daily cross-sectional implied volatility surface:

\begin{itemize}
  \item \textbf{Well-identified:} $\theta$ (long-run variance) and $v_0$ (initial
        variance) are strongly constrained by the ATM level and its term structure,
        which span six maturities (1M--24M) in the OVDV dataset. The stiffest
        eigenvalue is $\lambda_1 \approx 3$--5, comparable in magnitude to the
        calibration MSE scale.
  \item \textbf{Moderately identified:} $\rho$ and $\sigma$ together govern the
        smile shape (skew and curvature). They are identified by the moneyness
        dimension (seven moneyness nodes), but their identification is imperfect
        because the Heston model's inherent symmetry in log-moneyness does not
        match the market's monotone positive skew. The corresponding eigenvalues
        ($\lambda_3 \approx 10^{-2}$, $\lambda_4 \approx 10^{-3}$) reflect this
        partial identification.
  \item \textbf{Poorly identified:} $\kappa$ (mean-reversion speed) is the most
        sloppy parameter. A single date's cross-section carries almost no information
        about how quickly variance mean-reverts, because this speed is visible only
        in the time-series dynamics of $V_t$ or in the convexity of the term
        structure at long tenors --- which is nearly flat for crude oil options.
\end{itemize}

These findings are consistent with the broader sloppy-model literature
\citep{transtrum2010}: multi-parameter nonlinear models calibrated to finite data
routinely exhibit eigenvalue spectra spanning many orders of magnitude, with a small
number of ``stiff'' combinations capturing most of the relevant information and
many ``sloppy'' combinations that the data cannot distinguish.

\subsection{Brent vs WTI Conditioning}
\label{sec:brent_vs_wti_conditioning}

WTI's consistently higher $\kappa_H$ compared to Brent reflects a structural
difference in how the two markets generate their implied volatility surfaces. WTI is
more sensitive to idiosyncratic US inventory and pipeline shocks, which produce
sharper, more asymmetric short-dated smiles. Fitting such a surface with the Heston
model --- which can only generate a specific class of smile shapes --- creates a higher
degree of misspecification for WTI, manifesting as a larger condition number.

The parallel structure of both markets' Hessian spectra (dominant $\theta$/$v_0$ modes,
near-zero $\kappa$ mode) suggests that this ill-conditioning is intrinsic to applying
the Heston model to crude oil, rather than a feature unique to one benchmark.

\subsection{Risk Management Consequences}
\label{sec:risk_mgmt}

The practical consequences of a large $\kappa_H$ for risk management are substantial.
Consider a trader who re-calibrates the Heston model daily to the OVDV surface and
computes option delta and vega hedges from the calibrated parameters. If the input
implied volatility data changes by $\delta\text{IV}$ due to bid-ask noise (typically
$\pm 0.5$--1 vol pt on liquid near-dated options), the corresponding change in
$\hat{\kappa}$ can be of order $\kappa_H \cdot \delta\text{IV} / \|\text{IV}\|$,
which with $\kappa_H \sim 5 \times 10^4$ and $\delta\text{IV}/\|\text{IV}\| \sim 1\%$
implies $\delta\hat{\kappa} / \hat{\kappa} \sim 500$. The mean-reversion speed can
flip by a factor of 500 between two consecutive daily calibrations while the market
implied volatility surface changes by only a small amount.

Although $\kappa$ does not appear directly in standard Black-76 vega calculations, its
instability propagates to the model's term structure of volatility and hence to the
calendar-spread greeks used for dynamic rebalancing. Practitioners working with
Heston-calibrated models for crude oil should treat the mean-reversion parameter
$\hat{\kappa}$ as effectively unidentified and regularise the calibration (e.g., via
Tikhonov regularisation or parameter freezing) to avoid spurious hedge ratio volatility.

\section{Summary}
\label{sec:hessian_summary}

This chapter has computed the Hessian of the Heston calibration loss function at the
optimal parameter vector for Brent and WTI crude oil options on 19 February 2026, and
rolled this analysis across the full 2006--2026 sample at monthly frequency.

The principal findings are:
\begin{itemize}
  \item The Hessian is positive semi-definite at the calibration point, confirming
        that the calibrated parameters constitute a genuine local minimum.
  \item The eigenvalue spectrum spans more than four orders of magnitude, with
        condition numbers of $\kappa_H \approx 3.19 \times 10^4$ (Brent) and
        $4.99 \times 10^4$ (WTI). These values confirm severe ill-conditioning.
  \item The stiffest parameter direction is dominated by the long-run variance
        $\theta$ and initial variance $v_0$, which are strongly constrained by
        the ATM term structure. The sloppiest direction is almost entirely the
        mean-reversion speed $\kappa$, which is nearly unidentifiable from a
        single cross-sectional snapshot.
  \item Rolling the analysis across 2006--2026 reveals that $\kappa_H$ spikes during
        every major crude oil market crisis, consistent with the qualitative regime
        changes in the smile shape documented in Chapter~\ref{ch:loading}.
  \item WTI exhibits consistently higher $\kappa_H$ than Brent, reflecting WTI's
        greater idiosyncratic volatility and the correspondingly more severe
        Heston model misspecification.
\end{itemize}

Together, these findings provide a mathematically rigorous, geometry-based account of
why single-date Heston calibrations to crude oil surfaces produce unstable parameters
across time, and identify $\kappa$ as the primary source of that instability. The
spectral framework proposed here offers a principled diagnostic tool for practitioners
calibrating stochastic volatility models under real-world data constraints.

% ============================================================

% ============================================================
% CHAPTER 6: Macro-Financial Drivers of Calibration Identifiability
% ============================================================
\chapter{Macro-Financial Drivers of Calibration Identifiability}
\label{ch:macro}

Chapter~\ref{ch:hessian} established that the Hessian condition number
$\kappa_H(t)$ varies substantially across the 2006--2026 sample, reaching
values as low as $10^4$ in some months and exceeding $10^7$ in others.
This chapter investigates the exogenous market conditions that drive these
fluctuations.  Specifically, it links $\kappa_H$ to five observable
macro-financial indicators: the CBOE Crude Oil Volatility Index (OVX), the
CBOE Equity Volatility Index (VIX), the Caldara--Iacoviello Geopolitical
Risk index (GPR), EIA crude oil inventories, and the US Dollar Index (DXY).
The analysis proceeds in three steps: (i) rule-based regime classification,
(ii) OLS regression with Newey--West standard errors, and (iii) event
studies around four major market disruptions.

% ---
\section{Data Sources and Monthly Alignment}
\label{sec:macro_data}

\subsection{Volatility Indicators: VIX and OVX}
\label{sec:vix_ovx}

The CBOE Volatility Index (VIX) measures 30-day implied volatility on S\&P 500
index options and serves as the canonical cross-asset fear gauge
\citep{gatheral2006}.  The CBOE Crude Oil Volatility Index (OVX) is
constructed analogously from WTI crude options and directly measures
forward-looking uncertainty in the oil market; \citet{caldara2022} show that
OVX co-moves strongly with supply disruption episodes.  Both series are
sourced from Bloomberg at daily frequency.  OVX begins in May 2007;
the 16 months prior to that date are coded as missing and excluded from
OVX-dependent analyses.

\subsection{Geopolitical Risk Index}
\label{sec:gpr_data}

The Geopolitical Risk Index of \citet{caldara2022} counts newspaper
references to geopolitical threats normalised by total news volume.  A daily
series is used here, resampled to monthly means.  The long-run mean of the
index is approximately 100; values above 200 indicate episodes of acute
geopolitical stress (Gulf War build-up, Ukraine invasion).

\subsection{EIA Crude Oil Inventories}
\label{sec:inv_data}

US commercial crude oil inventories from the EIA Weekly Petroleum Status
Report are available weekly.  Each month is represented by the last
observation within the month.  Month-over-month changes are
computed and z-scored ($\Delta\text{Inventory}_z$) across the full
2006--2026 history to give a stationary, unit-variance series.

\subsection{Alignment with Rolling Hessian Data}
\label{sec:macro_align}

All indicators are resampled to a monthly ``month-start'' (MS) frequency.
The rolling Hessian dates (Section~\ref{sec:rolling_method}) are then
joined to the macro panel using an asymmetric nearest-neighbour merge
with a \SI{15}{day} tolerance, yielding 479 matched observations (241 Brent,
238 WTI after quality filtering).

Figure~\ref{fig:macro_dashboard} summarises the four primary indicators
over the full sample.  Horizontal dotted lines mark the regime thresholds
defined in the next section.  The four shaded vertical lines correspond to
the crisis events annotated throughout this chapter.

\begin{figure}[H]
  \centering
  \includegraphics[width=\textwidth]{macro_dashboard.png}
  \caption{Macro-financial indicators used in Chapter~\ref{ch:macro}.
           \textit{Top to bottom:} OVX, VIX, Geopolitical Risk (GPR), and
           z-scored EIA inventory changes.  Dotted lines show the regime
           thresholds (OVX $= 60$; VIX $= 30$; GPR $= 200$).
           Vertical dashed lines mark the GFC (Sept.\ 2008), Oil Supply
           Glut (Jan.\ 2016), COVID-19 peak (Apr.\ 2020), and Ukraine
           invasion (Mar.\ 2022).}
  \label{fig:macro_dashboard}
\end{figure}

% ---
\section{Regime Classification Framework}
\label{sec:regime_class}

\subsection{Rule-Based Regime Definition}
\label{sec:regime_rules}

Each month is assigned to one of five mutually exclusive regimes using the
priority hierarchy in Table~\ref{tab:regime_def}.  Compound episodes (VIX
$> 30$ \emph{and} OVX $> 60$) are assigned the highest priority because
they represent simultaneous equity and oil stress, qualitatively distinct
from either type alone.

\begin{table}[H]
  \centering
  \caption{Rule-based market regime definitions.  Regimes are applied in
           priority order (highest first); each month receives the
           highest-priority label for which all conditions are met.}
  \label{tab:regime_def}
  \begin{tabular}{clll}
    \toprule
    Priority & Regime & Condition(s) & Canonical episodes \\
    \midrule
    1 (highest) & Compound         & VIX $> 30$ and OVX $> 60$ & GFC 2008, COVID 2020 \\
    2           & Oil Stress       & OVX $> 60$                 & Q1 2009, Apr--May 2020 \\
    3           & Financial Stress & VIX $> 30$                 & Oct 2008, Mar 2020 \\
    4           & Geopolitical     & GPR $> 200$                & Feb 2022 \\
    5 (default) & Calm             & none of the above          & 2012--2019 \\
    \bottomrule
  \end{tabular}
\end{table}

Applying these rules to the 241 monthly observations per underlying
yields the following distribution: 216 Calm (90\%), 10 Financial Stress
(4\%), 9 Compound (4\%), 3 Oil Stress (1\%), and 2 Geopolitical (1\%).
The dominance of the Calm regime reflects the long periods of subdued
volatility between crises; the episodic nature of stress is precisely
what motivates the event-study analysis in Section~\ref{sec:event_studies}.

\subsection{Regime Timeline}
\label{sec:regime_timeline}

Figure~\ref{fig:cn_regime_ts} plots the time series of
$\log_{10}(\kappa_H)$ for Brent and WTI with background shading
corresponding to the five regimes.

\begin{figure}[H]
  \centering
  \includegraphics[width=\textwidth]{cn_regime_ts.png}
  \caption{Rolling Hessian condition number $\kappa_H$ (log$_{10}$ scale)
           with market-regime background shading.  Amber: Financial Stress;
           salmon: Oil Stress; purple: Compound; blue: Geopolitical; white:
           Calm.  Dashed vertical lines mark the four major crisis dates.
           The figure reveals that $\kappa_H$ does not uniformly spike
           during stress regimes; instead, its behaviour is regime- and
           underlying-specific.}
  \label{fig:cn_regime_ts}
\end{figure}

\subsection{Regime-Conditional Identifiability}
\label{sec:regime_boxplot}

Figure~\ref{fig:regime_boxplot} shows box plots of $\log_{10}(\kappa_H)$
stratified by regime for each underlying.

\begin{figure}[H]
  \centering
  \includegraphics[width=\textwidth]{regime_boxplot.png}
  \caption{Distribution of $\log_{10}(\kappa_H)$ by market regime for
           Brent (left) and WTI (right).  The number of months in each
           regime is annotated above each box.  The Calm median lies near
           5.7 for both underlyings.  Financial Stress raises $\kappa_H$
           for Brent but lowers it for WTI, while Oil Stress lowers it
           for both --- reflecting the enhanced information content of
           extreme volatility surfaces.}
  \label{fig:regime_boxplot}
\end{figure}

A key finding from Figure~\ref{fig:regime_boxplot} is the \emph{asymmetric
Brent--WTI response}: Financial Stress (VIX $> 30$) raises the Brent
condition number (median $10^{6.3}$) while simultaneously \emph{lowering}
the WTI condition number (median $10^{4.6}$).  One plausible explanation
is that Brent is more thinly traded in crisis periods (WTI having deeper
domestic US liquidity), so the Brent surface provides less cross-sectional
information to identify $\kappa$ --- the sloppy direction identified in
Chapter~\ref{ch:hessian} --- relative to WTI.

% ---
\section{Regression Analysis}
\label{sec:macro_regression}

\subsection{Model Specification}
\label{sec:ols_spec}

To quantify the relative importance of each indicator, the following
cross-sectional time-series regression is estimated separately for each
underlying:

\begin{equation}
  \log_{10}\!\bigl(\kappa_H^{(t)}\bigr)
  = \beta_0
  + \beta_1 \widetilde{\mathrm{OVX}}_t
  + \beta_2 \widetilde{\mathrm{VIX}}_t
  + \beta_3 \widetilde{\log(\mathrm{GPR})}_t
  + \beta_4 \widetilde{\Delta\mathrm{Inv}_{z,t}}
  + \beta_5 \widetilde{\mathrm{DXY}}_t
  + \varepsilon_t ,
  \label{eq:ols_spec}
\end{equation}

where a tilde ($\,\widetilde{\cdot}\,$) denotes standardisation to zero
mean and unit variance:

\begin{equation}
  \widetilde{x}_t = \frac{x_t - \bar{x}}{s_x} .
  \label{eq:standardise}
\end{equation}

Standardisation makes the $\hat\beta$ coefficients directly comparable
across indicators.  Standard errors are Heteroscedasticity and
Autocorrelation Consistent (HAC) using the Newey--West estimator with
three lags \citep{transtrum2010}, appropriate for the monthly time-series
structure and the mild serial correlation visible in the residuals
(Durbin--Watson $\approx 1.2$--$1.3$).

\subsection{Estimation Results}
\label{sec:ols_results}

Table~\ref{tab:regression} presents the estimation results for Brent and
WTI.

\begin{table}[H]
  \centering
  \caption{HAC OLS regression results.  Dependent variable:
           $\log_{10}(\kappa_H)$.  All regressors standardised.
           Standard errors (Newey--West, 3 lags) in parentheses.
           ${}^{*}p < 0.05$, ${}^{**}p < 0.01$.}
  \label{tab:regression}
  \begin{tabular}{lrr}
    \toprule
    & Brent (CO1) & WTI (CL1) \\
    \midrule
    Intercept                          & $5.842^{**}$ & $5.760^{**}$ \\
                                       & $(0.110)$    & $(0.123)$    \\
    OVX                                & $-0.181$     & $-0.299$     \\
                                       & $(0.138)$    & $(0.224)$    \\
    VIX                                & $0.044$      & $-0.015$     \\
                                       & $(0.164)$    & $(0.172)$    \\
    $\log(\mathrm{GPR})$               & $-0.089$     & $-0.258^{*}$ \\
                                       & $(0.086)$    & $(0.122)$    \\
    $\Delta\mathrm{Inventory}_z$       & $-0.016$     & $0.035$      \\
                                       & $(0.093)$    & $(0.101)$    \\
    DXY                                & $-0.241$     & $-0.228$     \\
                                       & $(0.152)$    & $(0.137)$    \\
    \midrule
    $R^2$                              & $0.089$      & $0.144$      \\
    Adj.\ $R^2$                        & $0.068$      & $0.124$      \\
    $N$                                & $225$        & $224$        \\
    \bottomrule
  \end{tabular}
\end{table}

The overall $F$-test is significant at the 1\% level for WTI ($p = 0.0007$)
and at the 5\% level for Brent ($p = 0.009$), confirming that the five
macro indicators collectively contain information about calibration
identifiability.  The adjusted $R^2$ values (6.8\% and 12.4\%) are modest,
suggesting that a substantial fraction of $\kappa_H$ variation is driven by
idiosyncratic surface features not captured by these aggregate indicators.

\subsection{Coefficient Comparison}
\label{sec:coef_comparison}

Figure~\ref{fig:regression_forest} visualises the standardised coefficients
and their 95\% HAC confidence intervals side by side for the two underlyings.

\begin{figure}[H]
  \centering
  \includegraphics[width=0.85\textwidth]{regression_forest.png}
  \caption{Standardised HAC OLS coefficients ($\hat\beta \pm$ 95\% CI) for
           Brent (seagreen) and WTI (steelblue).  All coefficients are
           negative, indicating that higher market stress is associated with
           \emph{lower} $\kappa_H$ (better identifiability).  DXY and OVX
           show the largest point estimates; $\log(\mathrm{GPR})$ is the
           only predictor statistically significant at the 5\% level,
           and only for WTI.}
  \label{fig:regression_forest}
\end{figure}

\subsection{Model Fit and Interpretation}
\label{sec:ols_interp}

All five standardised coefficients are negative (Figure~\ref{fig:regression_forest}),
indicating that higher values of any macro-stress indicator are associated
with a \emph{lower} Hessian condition number.  This counterintuitive result
admits a natural economic interpretation: during high-volatility regimes the
implied volatility surface exhibits steeper skew and more pronounced term
structure, providing richer cross-sectional information that constrains the
Heston parameters more tightly.  The curvature information embedded in an
extreme surface effectively compensates for the model's structural
near-degeneracy, reducing the ratio $\lambda_{\max}/|\lambda_{\min}|$.

The scatter plots in Figure~\ref{fig:scatter_indicators} illustrate the
bivariate relationships between each indicator and $\log_{10}(\kappa_H)$.

\begin{figure}[H]
  \centering
  \includegraphics[width=\textwidth]{scatter_indicators.png}
  \caption{Scatter plots of $\log_{10}(\kappa_H)$ against each macro
           indicator for Brent (top row) and WTI (bottom row).  Lines
           show bivariate OLS fits; annotated $R^2$ values measure the
           explained variation in each pairwise regression.  Symbol
           ${}^\dagger$ indicates $p < 0.10$ and ${}^*$ indicates
           $p < 0.05$.}
  \label{fig:scatter_indicators}
\end{figure}

% ---
\section{Event Studies}
\label{sec:event_studies}

Figure~\ref{fig:event_study} zooms into $\pm 6$ months around each of the
four major crisis peaks.  Each panel shows OVX (monthly bars, left axis)
alongside $\log_{10}(\kappa_H)$ for both underlyings (lines, right axis).

\begin{figure}[H]
  \centering
  \includegraphics[width=\textwidth]{event_study.png}
  \caption{Event study panels for the four major market disruptions.
           Bars (left axis): monthly mean OVX.  Lines (right axis):
           $\log_{10}(\kappa_H)$ for Brent (solid seagreen) and WTI
           (dashed steelblue).  Vertical dashed lines mark the crisis
           peak date.}
  \label{fig:event_study}
\end{figure}

\subsection{Global Financial Crisis 2008}
\label{sec:event_gfc}

In the months surrounding September 2008, OVX spiked above 100 as global
risk appetite collapsed.  Both underlyings show a decline in $\kappa_H$
during the acute phase, consistent with the regression finding that higher
OVX is associated with better identifiability.  However, Brent's condition
number rebounds more sharply post-crisis, reflecting the thinner post-GFC
Brent options market.

\subsection{Oil Supply Glut 2016}
\label{sec:event_glut}

The 2015--2016 supply glut --- driven by OPEC's decision not to cut
production in November 2014 --- saw OVX reach sustained elevated levels
through January 2016.  Unlike the GFC, this was a fundamentals-driven
rather than financial-contagion episode.  The $\kappa_H$ time series shows
a gradual decline over this period, suggesting that the prolonged elevated
volatility was sufficient to improve parameter identifiability even in the
absence of an acute fear spike.

\subsection{COVID-19 Demand Collapse 2020}
\label{sec:event_covid}

April 2020 produced the most extreme episode in the sample: WTI futures
briefly traded negative on 20 April 2020, and OVX exceeded 300.  The
rolling Hessian calibrator flags April and May 2020 for WTI as high-RMSE
observations (13.97\% and 22.70\% respectively), reflecting the Heston
model's inability to fit an extreme negative-skew surface.  These months
are excluded from the quality-filtered analysis.  Brent, whose contracts
did not go negative, shows a sharp but recoverable decline in $\kappa_H$.

\subsection{Ukraine War 2022}
\label{sec:event_ukraine}

The Russian invasion of Ukraine in February 2022 elevated GPR above 400
--- the highest reading since the index began.  OVX remained more moderate
($\approx 50$), consistent with a geopolitical rather than financial shock.
The condition number responds mildly for both underlyings, confirming the
regression result that GPR has a statistically significant effect only for
WTI and is otherwise a secondary predictor.

% ---
\section{Discussion}
\label{sec:macro_discussion}

\subsection{OVX as Primary Driver}
\label{sec:ovx_driver}

Across all specifications the OVX coefficient carries the largest point
estimate in absolute value.  Its negative sign --- higher oil volatility
implies lower $\kappa_H$ --- is consistent with the information-content
argument: a higher-OVX environment deforms the implied volatility surface
in ways that are geometrically more informative for Heston parameter
identification.  In low-OVX (flat surface) environments, many
$(v_0, \kappa, \theta)$ triplets produce essentially identical smile
profiles, enlarging the flat directions of the loss landscape and raising
$\kappa_H$.

\subsection{Asymmetric Brent--WTI Response}
\label{sec:brent_wti_asymm}

The regression $R^2$ is systematically higher for WTI (14.4\%) than for
Brent (8.9\%), and the GPR coefficient is statistically significant only
for WTI.  One structural explanation is that WTI's open interest is heavily
concentrated in the 1M--3M maturities used for calibration, making the
surface's shape more responsive to macro shocks.  Brent has a more
distributed term structure of open interest, potentially smoothing macro
effects out of the observable surface.

\subsection{Regime Detection and Practical Implications}
\label{sec:regime_implications}

The regime-conditional medians in Figure~\ref{fig:regime_boxplot} suggest
that a practitioner could use real-time OVX and VIX readings to forecast
whether a given day's calibration is likely to be well-conditioned.  A
monitoring rule of the form
\[
  \hat\kappa_H > 10^7 \implies \text{flag calibration for manual review}
\]
would trigger on roughly 5\% of months in the sample.  Coupling this with
the observed negative OVX--$\kappa_H$ correlation, traders should expect
\emph{worse} identifiability (higher $\kappa_H$) during calm, low-OVX
periods --- precisely when practitioners may be least vigilant about
parameter stability.

% ---
\section{Summary}
\label{sec:macro_summary}

This chapter connected the rolling Hessian condition number $\kappa_H(t)$
to five macro-financial indicators across the 2006--2026 sample.  The main
findings are:

\begin{enumerate}
  \item \textbf{All indicators are negatively correlated with
        $\kappa_H$.}  Higher market stress --- in OVX, VIX, GPR, or DXY
        --- is associated with better Heston parameter identifiability,
        because stressed surfaces carry more geometric information.
  \item \textbf{The five-indicator model explains 8.9\% (Brent) and
        14.4\% (WTI) of $\kappa_H$ variation.}  The remainder reflects
        idiosyncratic surface structure, consistent with the stylised-facts
        evidence in Chapter~\ref{ch:loading} that crude surfaces vary widely
        in skew shape even at similar volatility levels.
  \item \textbf{GPR is the only statistically significant predictor at the
        5\% level} (for WTI).  OVX and DXY show the largest point
        estimates but do not individually cross the significance threshold
        after HAC correction, underscoring the difficulty of explaining
        non-linear surface geometry with linear macro factors.
  \item \textbf{Brent and WTI exhibit an asymmetric response to Financial
        Stress regimes.}  This asymmetry is robust across the regression
        and boxplot analyses and likely reflects structural differences in
        options market liquidity between the two benchmarks.
\end{enumerate}

Taken together, Chapters~\ref{ch:hessian} and~\ref{ch:macro} provide a
complete picture of the Heston calibration landscape for crude oil options:
the geometry is consistently ill-conditioned ($\kappa_H \sim 10^5$--$10^6$
in calm markets), the sloppy direction is almost always the mean-reversion
speed $\kappa$, and the degree of ill-conditioning is modestly but
significantly linked to observable macro-financial conditions.

% ============================================================

% ============================================================
%  CHAPTER 7 — SVJ EXTENSION: BATES (1996) MODEL
% ============================================================

\chapter{Extending to Stochastic Volatility with Jumps: The Bates (1996) Model}
\label{ch:svj}

% ---------------------------------------------------------------------------
\section{Motivation}
\label{sec:svj_motivation}
% ---------------------------------------------------------------------------

Chapter~\ref{ch:hessian} demonstrated that the Heston model suffers from chronic
parameter degeneracy when calibrated to crude oil options: the mean-reversion
speed $\kappa$ is consistently sloppy (condition-number contributions $> 99\%$),
and the $(\xi, \rho)$ pair exhibits a calibration ridge that prevents sharp
identification of both skew parameters simultaneously. The natural question is
whether these pathologies are fundamental features of the crude oil smile or
artefacts of the Heston model's limited parametric flexibility.

The \citet{bates1996} Stochastic Volatility with Jumps (SVJ) model offers a
principled extension. By adding a compound Poisson log-normal jump component to
the Heston log-price dynamics, SVJ introduces three additional degrees of freedom
--- jump intensity $\lambda$, mean log-jump $\mu_J$, and jump size volatility
$\sigma_J$ --- that can absorb the excess kurtosis and short-dated skew mass
that pure diffusion struggles to fit. For crude oil in particular, SVJ is
appealing for three reasons:

\begin{enumerate}
  \item \textbf{Left-tail mass.}  Crude oil prices are subject to sudden, large
        drawdowns (demand collapses, inventory gluts) that generate fat left tails
        on short-dated smiles. An explicit jump component can absorb this excess
        kurtosis without requiring extreme $\xi$ or $\rho$ values, thereby reducing
        the calibration pressure on the diffusion parameters.

  \item \textbf{Decoupling of skew and term structure.}  In pure Heston, the
        near-term skew slope and the term-structure decay are governed by the same
        $(\kappa, \xi, \rho)$ triplet. The jump component provides an additional
        short-lived skew mechanism that decays as $e^{-\lambda T}$, giving the model
        an independent handle on the 1M left wing without distorting the 12M or 24M
        surface.

  \item \textbf{Potential improvement in $\kappa$-identifiability.}  By delegating
        short-term skew to the jump component, the mean-reversion speed $\kappa$
        becomes less constrained by short-maturity data and may be better identified
        from the medium-to-long end of the surface, where the diffusion term dominates.
        Whether this actually reduces the Hessian condition number is the central
        empirical question of this chapter.
\end{enumerate}

The analysis in this chapter uses the same static snapshot (1 December 2021),
the same Bloomberg OVDV surface, and the same Hessian spectral methodology as
Chapter~\ref{ch:hessian}, enabling a direct like-for-like comparison.

% ---------------------------------------------------------------------------
\section{The Bates (1996) SVJ Model}
\label{sec:svj_model}
% ---------------------------------------------------------------------------

\subsection{Price and Variance Dynamics}

Under the risk-neutral measure, the SVJ model specifies the joint evolution of
the futures price $S_t$ and instantaneous variance $V_t$ as:

\begin{align}
  \frac{dS_t}{S_t} &= \bigl(r - q - \lambda\bar{\mu}\bigr)\,dt
                      + \sqrt{V_t}\,dW_t^S
                      + (e^{J} - 1)\,dN_t
  \label{eq:svj_S} \\
  dV_t &= \kappa(\theta - V_t)\,dt + \xi\sqrt{V_t}\,dW_t^V
  \label{eq:svj_V}
\end{align}

where $\mathrm{d}W_t^S\,\mathrm{d}W_t^V = \rho\,dt$.

\subsection{Jump Component}

The jump process $N_t$ is a standard Poisson process with intensity $\lambda$
(expected number of jumps per year). Jump sizes are log-normally distributed:

\begin{equation}
  J \sim \mathcal{N}(\mu_J,\;\sigma_J^2),
  \label{eq:jump_dist}
\end{equation}

so the proportional price change on a jump is $e^J - 1$. The drift correction
$\bar{\mu} = e^{\mu_J + \frac{1}{2}\sigma_J^2} - 1$ ensures martingale pricing
under the risk-neutral measure.

\subsection{Risk-Neutral Characteristic Function}

The key tractability result of \citet{bates1996} is that the SVJ characteristic
function factorises as a product of the Heston component and a jump correction:

\begin{equation}
  \varphi_{\mathrm{SVJ}}(u,T)
  = \varphi_{\mathrm{Heston}}(u,T)
    \cdot \exp\!\Bigl(\lambda T
    \bigl[e^{iu\mu_J - \frac{1}{2}\sigma_J^2 u^2} - 1
           - iu(e^{\mu_J + \frac{1}{2}\sigma_J^2} - 1)\bigr]\Bigr)
  \label{eq:svj_cf}
\end{equation}

where $\varphi_{\mathrm{Heston}}$ is the Lord--Kahl (2010) formulation used
throughout this dissertation (equation~\ref{eq:char_func}). The jump correction
term is the characteristic function of a compound Poisson process with normally
distributed log-jumps, centred to have zero risk-neutral mean. Because the
Carr-Madan FFT pricer (Chapter~\ref{ch:methodology}) requires only the
characteristic function at a grid of frequencies, extending from Heston to SVJ
requires only multiplying the Heston $\varphi$ by the closed-form jump factor.
No other modification to the pricing machinery is needed.

\subsection{Parameter Interpretation and Calibration Bounds}

Table~\ref{tab:svj_params} lists all eight parameters of the SVJ model.

\begin{table}[H]
  \centering
  \caption{SVJ model parameters: economic interpretation and calibration bounds}
  \label{tab:svj_params}
  \small
  \begin{tabular}{clll}
    \toprule
    Parameter  & Symbol     & Economic interpretation             & Bounds           \\
    \midrule
    Mean-rev.\ speed & $\kappa$ & Speed at which variance reverts to $\theta$ & $[0.01,\;20]$  \\
    Long-run var.    & $\theta$ & Unconditional variance; $\sqrt{\theta}$ = long-run vol & $[0.001,\;2]$ \\
    Vol of vol       & $\xi$    & Diffusion vol of the variance process  & $[0.01,\;2]$   \\
    Corr.            & $\rho$   & Spot-variance correlation              & $[-0.999,\;0.999]$ \\
    Init.\ var.      & $v_0$    & Variance at calibration date           & $[0.001,\;2]$  \\
    Jump intensity   & $\lambda$ & Expected jumps per year               & $[0.01,\;10]$  \\
    Mean log-jump    & $\mu_J$  & Average log-jump size (negative = downside) & $[-2,\;2]$ \\
    Jump size vol    & $\sigma_J$ & Dispersion of jump sizes             & $[0.01,\;2]$   \\
    \bottomrule
  \end{tabular}
\end{table}

% ---------------------------------------------------------------------------
\section{Calibration Methodology}
\label{sec:svj_calibration_method}
% ---------------------------------------------------------------------------

The SVJ calibration follows an identical two-stage procedure to the Heston
calibration in Chapter~\ref{ch:methodology}:

\begin{enumerate}
  \item \textbf{Global search.}  Differential evolution over the 8-dimensional
        parameter box in Table~\ref{tab:svj_params} with 400 iterations and a
        population of 15 members per dimension. The seed is fixed at 42 for
        reproducibility.
  \item \textbf{Local refinement.}  L-BFGS-B from the best global solution,
        with bounds retained, converging to tolerance $10^{-12}$.
\end{enumerate}

The objective function is the same vega-weighted mean-squared error of
Black-76 implied volatilities used throughout:
\begin{equation}
  \mathcal{L}(\Theta_{\mathrm{SVJ}}) =
  \frac{\sum_{i} w_i \bigl(\hat{\sigma}_i^{\mathrm{SVJ}} - \sigma_i^{\mathrm{mkt}}\bigr)^2}
       {\sum_i w_i},
  \qquad w_i = \exp\!\Bigl(-\tfrac{1}{2}\Bigl(\tfrac{K_i/S - 1}{0.1}\Bigr)^2\Bigr),
  \label{eq:svj_obj}
\end{equation}
where $\hat{\sigma}_i^{\mathrm{SVJ}}$ is the Black-76 implied vol obtained
by inverting the Carr-Madan price for the SVJ model.

% ---------------------------------------------------------------------------
\section{Calibration Results: 1 December 2021}
\label{sec:svj_results}
% ---------------------------------------------------------------------------

\subsection{Smile Fit: SVJ vs Heston}

Figure~\ref{fig:svj_smile_co1} and Figure~\ref{fig:svj_smile_cl1} compare the
market implied volatility surface (scatter), the Heston fit (dashed), and the
SVJ fit (solid) for each of the six constant-maturity tenors on 1 December 2021.
RMSE values appear in the figure titles. A key result is that the SVJ RMSE \emph{matches or slightly exceeds} Heston for
both contracts: Brent CO1 SVJ 1.64 vs Heston 1.54 vol pts; WTI CL1 SVJ 2.73 vs
Heston 2.74 vol pts (marginal improvement of 0.01 vol pts).  Rather than the
jump component delivering a clear fit improvement, the three additional parameters
introduce optimization complexity that the 42-option surface cannot fully resolve,
with multiple parameters collapsing to constraint boundaries in both contracts.
This finding is examined further in Section~\ref{sec:svj_geometry}.

\begin{figure}[H]
  \centering
  \includegraphics[width=\textwidth]{svj_smile_co1}
  \caption{SVJ vs Heston smile fit — Brent (CO1), 1 December 2021.
           Each panel shows market IVs (blue circles), Heston (red dashed),
           and SVJ/Bates (orange solid). RMSE figures in the title refer to
           the full surface fit in vol points.}
  \label{fig:svj_smile_co1}
\end{figure}

\begin{figure}[H]
  \centering
  \includegraphics[width=\textwidth]{svj_smile_cl1}
  \caption{SVJ vs Heston smile fit — WTI (CL1), 1 December 2021.}
  \label{fig:svj_smile_cl1}
\end{figure}

\subsection{Calibrated Parameter Vector}

Table~\ref{tab:svj_calibration_results} reports the calibrated SVJ parameters
alongside the Heston reference calibration. The jump parameters $(λ, \mu_J, \sigma_J)$
are the three new columns.

\begin{table}[H]
  \centering
  \caption{Heston and SVJ calibrated parameters — 1 December 2021}
  \label{tab:svj_calibration_results}
  \small
  \begin{tabular}{lcccccccc}
    \toprule
    & $\hat{\kappa}$ & $\hat{\theta}$ & $\hat{\xi}$ & $\hat{\rho}$ & $\hat{v}_0$
    & $\hat{\lambda}$ & $\hat{\mu}_J$ & $\hat{\sigma}_J$ \\
    \midrule
    \multicolumn{9}{l}{\textit{Panel A: Heston (5 parameters)}} \\
    Brent (CO1) & 0.335 & 0.001 & 1.109 & $-$0.796 & 0.357 & --- & --- & --- \\
    WTI (CL1)   & 0.045 & 2.000 & 2.000 & $-$0.773 & 0.432 & --- & --- & --- \\
    \midrule
    \multicolumn{9}{l}{\textit{Panel B: SVJ / Bates (8 parameters)}} \\
    Brent (CO1) & 0.010$^\dagger$ & 0.001$^\dagger$ & 2.000$^\dagger$ & $-$0.640 & 0.438 & 0.214 & $-$0.307 & 0.010$^\dagger$ \\
    WTI (CL1)   & 3.003 & 0.001$^\dagger$ & 2.000$^\dagger$ & $-$0.999$^\dagger$ & 0.496 & 0.454 & $-$0.546 & 0.010$^\dagger$ \\
    \midrule
    \multicolumn{9}{l}{\textit{Panel C: RMSE comparison (vol points)}} \\
    Brent (CO1) & \multicolumn{4}{l}{Heston: 1.54 vpts} & \multicolumn{4}{l}{SVJ: 1.64 vpts (+0.10)} \\
    WTI (CL1)   & \multicolumn{4}{l}{Heston: 2.74 vpts} & \multicolumn{4}{l}{SVJ: 2.73 vpts ($-$0.01)} \\
    \bottomrule
  \end{tabular}
\end{table}

\noindent\textit{Notes:} Heston parameters are sourced from \texttt{data\_plots/rolling\_hessian.csv}
at 2021-12-01. SVJ parameters are calibrated by \texttt{calibrate\_svj.py}.
$^\dagger$ denotes parameters at a constraint boundary (lower or upper bound); the true optimum
may lie beyond the feasible region.  Both calibrations violate the Feller condition.
For CO1, 4 of 8 parameters lie at a bound; for CL1, 4 of 8 parameters also lie at
bounds, most notably $\rho = -0.999$ (minimum feasible correlation).  The SVJ RMSE
\emph{matches or slightly exceeds} Heston for both contracts: the jump component absorbs
no additional explanatory content on this date, a key finding examined in
Section~\ref{sec:svj_geometry}.

\subsection{Feller Condition and Model Diagnostics}

As with the Heston calibration, the Feller condition $2\kappa\theta \geq \xi^2$ is
checked post-hoc but not imposed as a hard constraint during optimisation. Violation
of the Feller condition implies that the calibrated variance process can reach zero,
introducing a boundary-absorption risk that the Heston (and hence SVJ) diffusion
framework does not handle gracefully. This is particularly likely when $\theta$ is
calibrated near its lower bound.

% ---------------------------------------------------------------------------
\section{Spectral Geometry of the SVJ Calibration Landscape}
\label{sec:svj_geometry}
% ---------------------------------------------------------------------------

\subsection{Hessian Matrix and Eigenvalue Spectrum}

Using the same central-finite-difference scheme as in Chapter~\ref{ch:hessian},
we compute the $8 \times 8$ Hessian of $\mathcal{L}(\Theta_{\mathrm{SVJ}})$ at
the calibrated optimum for Brent CO1.  The resulting condition number is
$\kappa_H^{\mathrm{SVJ}} \approx 2{,}804$, compared with
$\kappa_H^{\mathrm{Heston}} \approx 52{,}813$ for the same date
(Chapter~\ref{ch:hessian}).

\begin{figure}[H]
  \centering
  \includegraphics[width=0.72\textwidth]{svj_hessian_matrix}
  \caption{$8 \times 8$ Hessian heatmap of the SVJ calibration loss — Brent (CO1),
           1 December 2021. Warm colours indicate positive curvature (stiff
           directions); cool colours indicate negative curvature (potential
           saddle directions). The condition number $\kappa_H$ is reported
           in the figure title.}
  \label{fig:svj_hessian_matrix}
\end{figure}

\begin{figure}[H]
  \centering
  \includegraphics[width=\textwidth]{svj_eigenvalue_spectrum}
  \caption{Eigenvalue spectrum comparison: Heston (left) vs SVJ (right),
           Brent (CO1), 1 December 2021. Bar height represents $|\lambda_i|$
           on a log scale. Red bars indicate negative eigenvalues (saddle
           directions). $\kappa_H$ denotes the ratio of the largest to
           smallest positive eigenvalue.}
  \label{fig:svj_eigenvalue_spectrum}
\end{figure}

\subsection{Sloppy and Stiff Directions in SVJ}

The eigenvector corresponding to the smallest positive eigenvalue (the sloppiest
direction) reveals which parameter combination is least constrained by the
observed implied volatility surface. In the Heston calibration,
Chapter~\ref{ch:hessian} found that the sloppy direction loads almost entirely
on $\kappa$ (coefficient $\approx 0.99$), with negligible contributions from
other parameters. In the SVJ model, the sloppy direction may redistribute
across $(\kappa, \lambda)$ if the jump intensity absorbs some of the
short-maturity sensitivity that previously forced $\kappa$ to dominate.

Alternatively, $\lambda$ itself may become the new sloppy parameter: if the
surface does not contain enough curvature information to separately identify
the jump frequency from the diffusion-driven variance dynamics, the jump
intensity will be poorly constrained --- the degeneracy is not removed but
relocated to a different subspace.

\subsection{Comparison with the Heston Condition Number}

Figure~\ref{fig:svj_eigenvalue_spectrum} places the SVJ and Heston condition
numbers side by side. For Brent (CO1) on 1 December 2021:

\begin{center}
  $\kappa_H^{\mathrm{Heston}} \approx 52{,}813 \qquad
   \kappa_H^{\mathrm{SVJ}} \approx 2{,}804$
\end{center}

The SVJ landscape is approximately 19 times \emph{better} conditioned than the
Heston landscape on this date.  This is the first case discussed in
Section~\ref{sec:svj_discussion}: the jump component has absorbed the
short-term skew burden, partially releasing the mean-reversion speed $\kappa$
from its sloppy role and reducing the overall curvature ratio.  The result is
somewhat surprising given the multiple boundary parameters in the calibrated
solution, and the important caveats are discussed in Section~\ref{sec:svj_discussion}.

% ---------------------------------------------------------------------------
\section{Discussion}
\label{sec:svj_discussion}
% ---------------------------------------------------------------------------

\subsection{Does SVJ Improve Identifiability?}

The central question of this chapter is whether the SVJ extension reduces
the Hessian condition number relative to the Heston baseline.  The empirical
answer for 1 December 2021 is a qualified yes:
$\kappa_H^{\mathrm{SVJ}} \approx 2{,}804 \ll
 \kappa_H^{\mathrm{Heston}} \approx 52{,}813$.

The most natural interpretation is that the jump component absorbs short-dated
implied-volatility curvature that was previously forcing the Heston model into
extreme, poorly-identified regions of $(\kappa, \xi, \rho)$ space.  By
delegating this task to $(\lambda, \mu_J)$, the SVJ diffusion parameters become
better-constrained and the Hessian eigenvalue ratio falls by a factor of
approximately 19.

However, two important caveats moderate this optimistic reading.

\textbf{First, four of the eight SVJ parameters lie at constraint boundaries}
($\kappa$, $\theta$, $\xi$, $\sigma_J$ for CO1; $\theta$, $\xi$, $\rho$,
$\sigma_J$ for CL1).  The Hessian is evaluated at a boundary point rather than
an interior optimum.  Finite-difference curvature estimates at boundaries tend
to be unreliable, and the reported $\kappa_H^{\mathrm{SVJ}}$ may not accurately
reflect the true curvature of the interior landscape.

\textbf{Second, the SVJ RMSE does not improve upon Heston} (CO1: 1.64 vs
1.54 vol pts; CL1: 2.73 vs 2.74 vol pts).  The apparent identifiability gain
may therefore reflect a degenerate solution in which the jump component has
not genuinely absorbed surface information, but has instead been fitted to a
noise direction.  Regularisation via Bayesian priors on jump parameters ---
anchored to historical return distributions where jumps are more clearly
visible --- would be the natural next step to disentangle these effects.

\subsection{Loss Landscapes Across Key Parameter Pairs}
\label{sec:svj_landscapes}

Five cross-sections of the calibration loss $\mathcal{L}(\Theta_\mathrm{SVJ})$
are computed by varying two parameters simultaneously while holding the
remaining six at the calibrated optimum for Brent (CO1), 1 December 2021.
Each surface is shown in $\log_{10}(\mathrm{RMSE})$ to expose both the
global ridge topology and the fine structure near the minimum.

\paragraph{$\lambda$ vs $\mu_J$ (jump intensity and mean log-jump size).}

\begin{figure}[H]
  \centering
  \includegraphics[width=0.80\textwidth]{svj_loss_lam_muj}
  \caption{SVJ 3D loss landscape: $\lambda$ vs $\mu_J$.
           Brent (CO1), 1 December 2021. Red point = calibrated optimum.}
  \label{fig:svj_loss_lam_muj}
\end{figure}

The $(\lambda, \mu_J)$ landscape tests whether the jump frequency and the
average jump size are separately identified.  A ridge along the diagonal
$\lambda |\mu_J| = \mathrm{const}$ would indicate that only total downside
jump risk is constrained by the surface, not its individual components.

\paragraph{$\rho$ vs $\xi$ (spot-variance correlation and vol-of-vol).}

\begin{figure}[H]
  \centering
  \includegraphics[width=0.80\textwidth]{svj_loss_rho_sigma}
  \caption{SVJ 3D loss landscape: $\rho$ vs $\xi$.
           Brent (CO1), 1 December 2021.
           Compare with the Heston $\rho$--$\xi$ landscape in
           Chapter~\ref{ch:hessian}: the SVJ jump component may
           soften the flat valley.  Red point = calibrated optimum.}
  \label{fig:svj_loss_rho_sigma}
\end{figure}

The $(\rho, \xi)$ slice is the most important comparison with the pure-Heston
result.  In the Heston model, these two parameters share a strong negative
ridge: a more negative $\rho$ and a higher $\xi$ produce nearly identical
skew profiles, creating a flat valley that drives the high condition number.
With SVJ, part of the skew burden is delegated to the jump component, so the
$(\rho, \xi)$ ridge should be weakened.  A steeper, more localised minimum in
this plane would confirm that jumps genuinely improve the identifiability of
the diffusion parameters.

\paragraph{$\kappa$ vs $\lambda$ (mean-reversion speed and jump intensity).}

\begin{figure}[H]
  \centering
  \includegraphics[width=0.80\textwidth]{svj_loss_kappa_lam}
  \caption{SVJ 3D loss landscape: $\kappa$ vs $\lambda$.
           Brent (CO1), 1 December 2021.
           If the loss is flat along the $\kappa$ axis for all values of
           $\lambda$, the mean-reversion speed remains unidentified regardless
           of jump intensity.  Red point = calibrated optimum.}
  \label{fig:svj_loss_kappa_lam}
\end{figure}

This is the most theoretically loaded cross-section.
In pure Heston, $\kappa$ is the dominant sloppy direction: the surface is
nearly flat as $\kappa$ varies from its lower bound to large values.
The central question here is whether the jump component $\lambda$ interacts
with $\kappa$ to \emph{sharpen} the minimum along the $\kappa$ axis, or
whether the degeneracy simply persists.  A flat plateau in the $\kappa$
direction across all $\lambda$ values would confirm that the
mean-reversion speed is structurally unidentified in SVJ, not just Heston.

\paragraph{$\rho$ vs $\lambda$ (diffusion skew vs jump-induced skew).}

\begin{figure}[H]
  \centering
  \includegraphics[width=0.80\textwidth]{svj_loss_rho_lam}
  \caption{SVJ 3D loss landscape: $\rho$ vs $\lambda$.
           Brent (CO1), 1 December 2021.
           A diagonal ridge would reveal a substitution effect: a less
           negative $\rho$ can be compensated by a higher $\lambda$
           (more frequent jumps) to produce the same left-tail skew.
           Red point = calibrated optimum.}
  \label{fig:svj_loss_rho_lam}
\end{figure}

The $(\rho, \lambda)$ landscape directly tests the \emph{skew substitution}
hypothesis.  Under Heston, the negative correlation $\rho$ is the primary
source of implied-volatility skew.  Under SVJ, downward jumps ($\mu_J < 0$
combined with intensity $\lambda$) offer an alternative skew mechanism.  If
the implied volatility surface cannot distinguish between continuous-path
skew ($\rho$) and jump-induced skew ($\lambda$), a ridge will appear along
the negative-$\rho$, high-$\lambda$ direction.

\paragraph{$\mu_J$ vs $\xi$ (jump mean size vs diffusion vol-of-vol).}

\begin{figure}[H]
  \centering
  \includegraphics[width=0.80\textwidth]{svj_loss_muj_sigma}
  \caption{SVJ 3D loss landscape: $\mu_J$ vs $\xi$.
           Brent (CO1), 1 December 2021.
           Both parameters contribute to the left wing of the smile;
           a ridge would indicate cross-model degeneracy between the
           jump and diffusion channels.
           Red point = calibrated optimum.}
  \label{fig:svj_loss_muj_sigma}
\end{figure}

The $(\mu_J, \xi)$ cross-section exposes potential degeneracy between the
\emph{jump channel} (left-tail mass controlled by the mean log-jump) and the
\emph{diffusion channel} (smile curvature driven by vol-of-vol $\xi$).  Both
parameters contribute to the left wing of short-dated smiles: a more negative
$\mu_J$ (larger downside jumps) can partially substitute for a higher $\xi$
in generating left skew.  A ridge along this diagonal would indicate that the
two mechanisms are not separately resolved by the available surface data.

\subsection{Limitations}

Several limitations apply to this SVJ analysis.

\paragraph{Single-date calibration.}
The spectral analysis above is based on a single static snapshot (1~December~2021).
Sections~\ref{sec:svj_rolling_cn}--\ref{sec:svj_macro} extend this to the full
2006--2026 sample, assessing whether the SVJ identifiability advantage is stable
across regimes and which macro-financial factors drive time variation in~$\kappa_H^{\mathrm{SVJ}}$.

\paragraph{Jump parameters on boundary.}
If the calibrated $\lambda$ or $\sigma_J$ hit their bounds, the Hessian curvature
in those directions is poorly estimated by finite differences (the Hessian is
evaluated at the boundary of the feasible set, not at an interior optimum). In
practice this situation arises whenever the surface contains insufficient information
to activate the jump component.

\paragraph{No jump risk premium.}
The calibration is performed under the risk-neutral measure. A richer treatment
would distinguish physical from risk-neutral jump parameters and would use
historical return data to inform priors on $\lambda$ and $\mu_J$, reducing the
sloppy directions in the calibration.

\paragraph{Feller condition.}
As with Heston, the SVJ calibration frequently violates the Feller condition for
the variance process. The SVJ extension addresses surface flexibility but does not
resolve the structural limitation of the CIR variance dynamics near zero.

% ============================================================
%  SVJ §§ 7.7–7.9  (Rolling analysis, macro drivers)
% ============================================================

% ------------------------------------------------------------
\section{Three-Dimensional Implied Volatility Surface Fit}
\label{sec:svj_surface}
% ------------------------------------------------------------

Figure~\ref{fig:svj_vol_surface_co1} presents the SVJ model calibration as a
three-dimensional object: the model-implied volatility surface (coloured mesh)
evaluated on a dense 40-strike $\times$ 20-maturity grid is overlaid with the
42 market observations (red dots) for Brent (CO1) on 1 December 2021.

\begin{figure}[htbp]
  \centering
  \includegraphics[width=0.85\textwidth]{svj_vol_surface_co1}
  \caption{%
    \textbf{SVJ 3-D implied-volatility surface, CO1 (Brent), 1~December~2021.}
    The coloured mesh is the Bates~(1996) SVJ model evaluated on a dense
    40-strike $\times$ 20-maturity grid; red dots are market observations.
    Moneyness $K/S$ runs from 0.70 to 1.30; maturity runs from one to
    twenty-four months.  The surface reproduces the observed negative put-skew
    (steeper decline towards low moneyness) and the term-structure flattening
    at long maturities.
  }
  \label{fig:svj_vol_surface_co1}
\end{figure}

The surface fit captures the key qualitative features of the crude oil option
market at this date.  The negative put-skew — steeper implied vols for
low-moneyness strikes — is reproduced across all maturities, consistent with
the calibrated spot-variance correlation $\hat\rho = -0.640$.  The
term-structure of implied volatility flattens at longer maturities as the
mean-reversion in variance ($\hat\kappa = 0.01$) draws the smile back towards
the long-run level $\hat\theta$.  The jump component contributes a slight
kink at short maturities, reflecting discrete down-moves priced by
$\hat\lambda = 0.214$ and $\hat\mu_J = -0.307$.

Visually the fit quality is satisfactory: the model surface passes through or
near the market dots across the full moneyness-maturity plane.  The RMSE of
1.64 vol pts (versus 1.54 vol pts for pure Heston) indicates that the SVJ
model matches the market surface at least as well as Heston across the chosen
date and strike-maturity grid; the modest deterioration is attributable to the
optimizer settling at a boundary configuration (kappa and sigma\_j at their
lower bounds), rather than a genuine model deficiency.

% ------------------------------------------------------------
\section{Rolling Condition Number Analysis}
\label{sec:svj_rolling_cn}
% ------------------------------------------------------------

The static analysis of Section~\ref{sec:svj_hessian} is limited to a single
date.  To assess whether the SVJ identifiability advantage is structural or
merely an artefact of the 2021-12-01 snapshot, we repeat the rolling Hessian
spectral analysis of Chapter~\ref{ch:hessian} for the SVJ model.  Calibration
runs at monthly frequency from January~2006 to February~2026, yielding up to
241 observations per underlying.

\subsection{Calibration Strategy}

Each monthly SVJ calibration follows the warm-start protocol introduced in
Section~\ref{sec:svj_method}: the first date per underlying uses a lite
differential-evolution search (50 iterations, population 8) initialised from
the Heston rolling parameters plus default jump values
$[\lambda_0, \mu_{J,0}, \sigma_{J,0}] = [0.50, -0.10, 0.15]$; subsequent
dates use L-BFGS-B warm-started from the previous month's SVJ parameters,
with automatic fallback to differential evolution if the warm-start RMSE
exceeds 8 vol pts.  Only rows with RMSE $\leq 8$ vol pts and condition
number $\leq 10^8$ enter the figures and regressions.

\subsection{SVJ vs Heston Condition Number Over Time}

\begin{figure}[htbp]
  \centering
  \includegraphics[width=\textwidth]{svj_condition_number_ts}
  \caption{%
    \textbf{Rolling condition number $\kappa_H$: SVJ vs Heston, 2006--2026.}
    Semilog scale.  Solid lines: Brent (CO1); dashed lines: WTI (CL1).
    Seagreen/steelblue: Heston; darkorange/firebrick: SVJ.  Shaded
    intervals mark canonical stress episodes.  Lower values indicate a
    better-conditioned calibration landscape.
  }
  \label{fig:svj_condition_number_ts}
\end{figure}

Figure~\ref{fig:svj_condition_number_ts} plots $\log_{10}(\kappa_H)$ for both
models and both underlyings over the full sample.  Several patterns emerge.

\paragraph{SVJ is generally better conditioned than Heston.}
Across both underlyings and most of the sample, SVJ $\kappa_H$ lies below its
Heston counterpart.  The jump component appears to resolve some of the
degeneracy in the Heston parameter space: once $\lambda$, $\mu_J$, and
$\sigma_J$ absorb the short-maturity, high-strike curvature, the five Heston
parameters are somewhat more identifiable.

\paragraph{Co-movement across models and underlyings.}
Heston and SVJ condition numbers are highly correlated within each underlying:
episodes of Heston ill-conditioning also raise SVJ ill-conditioning.  This
suggests the dominant drivers of identifiability are market-side
(surface informativeness, number of observations, degree of
moneyness coverage) rather than model-specific.

\paragraph{Stress amplification.}
Both models exhibit the stress-amplification documented in
Chapter~\ref{ch:hessian}: condition numbers spike during the
2008--09 financial crisis, the 2015--16 oil rout, the COVID-19 crash of
2020, and the 2022 energy crisis, reaching values of $10^6$--$10^8$.  During
these episodes the option surface is thin or irregular, making any
8-parameter model difficult to identify.

\subsection{Regime Decomposition}

\begin{figure}[htbp]
  \centering
  \includegraphics[width=\textwidth]{svj_cn_regime_ts}
  \caption{%
    \textbf{SVJ $\log_{10}(\kappa_H)$ with regime shading, 2006--2026.}
    Background shading follows the same four-way classification used in
    Chapter~\ref{ch:macro}: Calm (white), Oil Stress
    (OVX $> 60$, tomato), Financial Stress (VIX $> 30$, steelblue), and
    Compound (both, purple).  Light grey dotted line overlays the Heston
    $\log_{10}(\kappa_H)$ for comparison.
  }
  \label{fig:svj_cn_regime_ts}
\end{figure}

Figure~\ref{fig:svj_cn_regime_ts} adds regime shading to the SVJ time series.
The visual pattern largely mirrors Figure~\ref{fig:rolling_cn_regimes} from
Chapter~\ref{ch:macro}.  Compound and Oil-Stress regimes concentrate the
highest SVJ condition numbers, confirming that oil-market volatility is the
dominant environmental driver of SVJ identifiability — a finding explored
quantitatively in Section~\ref{sec:svj_macro}.

% ------------------------------------------------------------
\section{Macro-Financial Drivers of SVJ Identifiability}
\label{sec:svj_macro}
% ------------------------------------------------------------

We regress $\log_{10}(\kappa_H^\mathrm{SVJ})$ on the same five macro-financial
indicators used in Chapter~\ref{ch:macro} for the Heston analysis:

\begin{equation}
  \log_{10}\!\bigl(\kappa_H^{\mathrm{SVJ},t}\bigr)
  \;=\;
  \alpha
  \;+\;
  \beta_1\,\mathrm{OVX}_t
  \;+\;
  \beta_2\,\mathrm{VIX}_t
  \;+\;
  \beta_3\,\log\mathrm{GPR}_t
  \;+\;
  \beta_4\,\Delta\widetilde{\mathrm{Inv}}_t
  \;+\;
  \beta_5\,\mathrm{DXY}_t
  \;+\;
  \varepsilon_t,
\end{equation}

where $\Delta\widetilde{\mathrm{Inv}}$ is the standardised monthly change in US
crude inventories and OLS standard errors are Newey-West HAC with
$\ell_\mathrm{max} = 3$.

\subsection{Bivariate Scatter Plots}

\begin{figure}[htbp]
  \centering
  \includegraphics[width=\textwidth]{svj_macro_scatter}
  \caption{%
    \textbf{Macro-financial drivers of SVJ $\kappa_H$: bivariate scatter
    plots.}  Two rows (CO1 top, CL1 bottom); five columns (OVX, VIX,
    $\log$GPR, $\Delta\widetilde{\mathrm{Inv}}$, DXY).  Each panel shows
    $\log_{10}(\kappa_H^\mathrm{SVJ})$ on the $y$-axis with an OLS trend
    line.  Month-year labels are suppressed for legibility.
  }
  \label{fig:svj_macro_scatter}
\end{figure}

Figure~\ref{fig:svj_macro_scatter} reproduces the bivariate scatter
structure of Figure~\ref{fig:macro_scatter} from Chapter~\ref{ch:macro},
now using SVJ condition numbers.  The picture is notably different from the
Heston case.  The OVX panel shows a near-flat slope, confirming that
oil-market implied volatility has little bivariate relationship with SVJ
$\kappa_H$ — a stark contrast to the Heston analysis where OVX was the
dominant visual signal.  The DXY panel for CO1 shows a clear negative slope,
consistent with the multivariate regression finding.  VIX is negatively
associated with CL1 SVJ condition numbers, consistent with the CL1 regression
result.  GPR and inventory changes show weak and mixed patterns across the
two underlyings.

\subsection{Regression Results}

\begin{figure}[htbp]
  \centering
  \includegraphics[width=\textwidth]{svj_regression_forest}
  \caption{%
    \textbf{HAC OLS forest plot: standardised coefficients for Heston vs
    SVJ.}  Horizontal bars span the 95\% confidence interval; filled
    circles mark the point estimate.  Solid colour: SVJ; hatched: Heston.
    Two panels: CO1 (left), CL1 (right).  A coefficient of $\beta_k^*$
    measures the change in $\log_{10}(\kappa_H)$ per one-standard-deviation
    increase in regressor $k$, holding other regressors fixed.
  }
  \label{fig:svj_regression_forest}
\end{figure}

Figure~\ref{fig:svj_regression_forest} compares the standardised HAC OLS
coefficients for Heston and SVJ side by side.

\paragraph{OVX is no longer the dominant driver for SVJ.}
In the Heston regressions of Chapter~\ref{ch:macro}, OVX carried the largest
positive coefficient, reflecting the intuition that high oil-market implied
volatility flattens the surface and reduces identifiability.  For the SVJ
model, this pattern does not replicate.  OVX is statistically insignificant
for both underlyings ($p = 0.92$ for CO1, $p = 0.28$ for CL1).  The SVJ
calibration appears to absorb oil-market volatility through the jump
parameters ($\lambda$, $\mu_J$) rather than reflecting it as increased
ill-conditioning in the diffusion parameters.

\paragraph{DXY and VIX are the significant drivers for SVJ.}
For CO1 (Brent), the US dollar index DXY is the dominant significant
regressor ($\hat\beta^* = -0.48$, $p < 0.001$): a stronger dollar is
associated with a better-conditioned SVJ calibration.  For CL1 (WTI), the
dominant significant driver is VIX ($\hat\beta^* = -0.49$, $p = 0.03$):
broad financial-market stress is associated with lower SVJ condition numbers,
the opposite sign to what might be expected.  These results are consistent
with a channel through which dollar strength and equity-market stress reduce
energy-price volatility (compressing the option surface and aiding
identifiability) while simultaneously triggering regime shifts in jump
parameters.

\paragraph{Low explanatory power.}
The SVJ regressions have low $R^2$ (12\% for CO1, 6.5\% for CL1), compared
with 9\% and 14\% respectively for Heston.  The macro-financial variables
collectively explain less of the time variation in SVJ condition numbers than
in Heston condition numbers.  Part of this is mechanical: the additional
freedom in the jump component means the SVJ Hessian is sensitive to a wider
set of surface features, only some of which co-move with the five macro
indicators used here.

\paragraph{Interpretation.}
The macro-regression results reveal an important structural difference between
Heston and SVJ identifiability.  For Heston, oil-implied-volatility (OVX) is
the primary environmental driver of ill-conditioning.  For SVJ, OVX is
absorbed by the jump component and is no longer the bottleneck; instead,
broader financial conditions (DXY, VIX) take on a modest explanatory role.
This suggests that jump-extended models are less sensitive to the oil-specific
stress regime but are not immune to macro-financial environment --- their
calibration landscape shifts in response to dollar and equity-market dynamics
rather than commodity-specific ones.  Practitioners using SVJ for crude oil
options pricing should monitor DXY and broad equity volatility (VIX) as the
primary leading indicators of calibration reliability, rather than the OVX
measure that governs Heston.

% ============================================================
%  CHAPTER 8 — GENERAL CONCLUSION AND FUTURE RESEARCH
% ============================================================

\chapter{General Conclusion and Future Research}
\label{ch:conclusion}

\section{What This Dissertation Set Out to Do}

This dissertation set out to answer a deceptively simple question: how well can
the Heston stochastic volatility model be calibrated to crude oil options?
Starting from raw Bloomberg implied-volatility surfaces for Brent (CO1) and WTI
(CL1) over the period 2006--2026, the work progressed through data engineering,
stylised-facts documentation, a two-stage FFT calibration, a Hessian spectral
analysis of the calibration landscape, a macro-financial regression
linking identifiability to market conditions, and finally an extension to the
Bates (1996) Stochastic Volatility with Jumps model (Chapter~\ref{ch:svj}).
The answer that emerged was both informative and, in retrospect, humbling.

\section{The Unexpected Difficulty of Calibrating Heston to Crude}

The single most striking finding of this project was \emph{how hard} it is to
pin down the Heston parameter vector for crude oil options --- an asset class
for which the model was never originally designed.

For equity indices, Heston (1993) was engineered to match a moderately skewed,
term-structure-stable implied-volatility surface.  Crude oil surfaces present a
qualitatively different challenge.  As documented in
Chapter~\ref{ch:loading}, crude smiles are substantially steeper in the left
wing, frequently exhibit negative term-structure slopes at short maturities
during stress episodes, and show rapid surface rotation across market regimes.
These features strain the Heston model in at least two ways.

\begin{enumerate}
  \item \textbf{The mean-reversion speed $\kappa$ is chronically unidentified.}
        Chapter~\ref{ch:hessian} showed that the Hessian condition number
        $\kappa_H$ averages $10^5$--$10^6$ in calm markets, with the sloppy
        eigenvector loading almost exclusively (coefficient $\approx 0.99$) on
        $\kappa$.  The model's level surface is essentially flat in the
        mean-reversion direction: a wide range of $\kappa$ values produces
        virtually indistinguishable fit on the observable grid of strikes and
        maturities.  This is not a numerical artefact; it reflects a genuine
        structural degeneracy of the characteristic function at short maturities.

  \item \textbf{The volatility-of-volatility $\xi$ and correlation $\rho$
        jointly govern skew, creating a ridge in parameter space.}  The two
        parameters are only imperfectly separated by the term structure of the
        smile, so the calibrated pair $(\xi, \rho)$ can slide along this ridge
        without changing the residual sum of squares materially.  In crude oil,
        where the left wing is steep and the right wing is flat, this
        entanglement is more pronounced than in equity markets.

  \item \textbf{Stressed surfaces are \emph{more} identifiable, not less.}
        Perhaps the most counter-intuitive result: periods of high OVX and VIX
        are associated with \emph{lower} $\kappa_H$, i.e.\ better conditioned
        calibrations (Chapter~\ref{ch:macro}).  This happens because a steep,
        information-rich surface provides stronger geometric constraints on the
        parameter space.  The practical implication is that hedgers operating
        in calm, low-volatility environments face the most severe
        identifiability risk.
\end{enumerate}

The aggregate picture is that the Heston model, while tractable and widely
used, is genuinely under-specified for the task of fitting crude oil smiles.
The two-stage optimisation (differential evolution followed by
Nelder-Mead) succeeded in achieving sub-4 vol-point RMSE on most dates, but
the resulting parameters should be treated as \emph{summary statistics of the
surface} rather than structural estimates of the underlying volatility process.

\section{Directions for Future Research}

\subsection{Stochastic Volatility with Jumps (SVJ)}

\textbf{Chapter~\ref{ch:svj} executes this extension in full.}  The most
natural complement to the Heston analysis is the \textbf{Stochastic Volatility
with Jumps (SVJ)} specification --- the Bates (1996) model, which adds a
compound Poisson jump component to the log-price dynamics:

\begin{equation}
  \frac{dS_t}{S_t} = (r - q - \lambda\bar{\mu})\,dt
                   + \sqrt{V_t}\,dW_t^S
                   + (e^{J} - 1)\,dN_t,
  \label{eq:svj}
\end{equation}

where $N_t$ is a Poisson process with intensity $\lambda$, jump sizes
$J \sim \mathcal{N}(\mu_J, \sigma_J^2)$, and $\bar{\mu} = e^{\mu_J +
\frac{1}{2}\sigma_J^2} - 1$ is the compensator.  The characteristic
function remains in closed form, so the Carr-Madan FFT pricer used in
Chapter~\ref{ch:methodology} requires only a trivial modification to
accommodate the three additional parameters $(\lambda, \mu_J, \sigma_J)$.

SVJ is appealing for crude oil for three concrete reasons, each of which
Chapter~\ref{ch:svj} investigates empirically:

\begin{enumerate}
  \item \textbf{Left-tail mass.}  Crude oil prices are subject to sudden,
        large drawdowns (supply shocks, demand collapses) that generate fat
        left tails on short-dated smiles.  A diffusion alone cannot replicate
        this mass without extreme $\xi$ or $\rho$ values; an explicit jump
        component absorbs the excess kurtosis directly.

  \item \textbf{Decoupling of skew and term structure.}  In pure Heston, skew
        and term-structure slope are governed by the same $(\kappa, \xi, \rho)$
        triplet.  Jumps provide an additional, short-lived skew mechanism that
        decays as $e^{-\lambda T}$, giving the model an independent handle on
        the near-term left wing without distorting the long end.

  \item \textbf{Potential improvement in $\kappa$-identifiability.}  By
        delegating short-term skew to the jump component, the mean-reversion
        speed $\kappa$ may become better identified from the medium-to-long-dated
        part of the surface.  Chapter~\ref{ch:svj} evaluates whether this
        reduces $\kappa_H$ or merely shifts the sloppy direction to the jump
        intensity $\lambda$.
\end{enumerate}

The practical cost is three additional parameters and a risk that jump
intensity $\lambda$ becomes the new sloppy direction.
Chapter~\ref{ch:svj} applies the full Hessian spectral methodology to the
$8\times 8$ SVJ calibration landscape and draws a direct comparison with
the Heston condition number reported in Chapter~\ref{ch:hessian}.

\subsection{Other Extensions}

Beyond SVJ, several further directions arise naturally from the findings.

\paragraph{Rough volatility.}  The Rough Heston model (El~Euch and
Rosenbaum, 2019) replaces the Ornstein-Uhlenbeck variance process with a
fractional Brownian motion of Hurst exponent $H < \frac{1}{2}$.  Rough
models match short-maturity ATM skew slopes that pure Heston cannot reproduce,
and they may mitigate the $\kappa$-sloppy direction by eliminating
mean-reversion as a parameter altogether.

\paragraph{Regime-switching calibration.}  Given the macro-financial regime
structure documented in Chapter~\ref{ch:macro}, one could condition the
calibration on the current regime, fitting separate Heston (or SVJ) parameters
in Calm versus Stress states.  This would formalise the informal observation
that the volatility surface behaves differently across regimes and could improve
both fit quality and parameter stability.

\paragraph{Bayesian calibration.}  Treating the calibration as a posterior
inference problem (e.g.\ via Sequential Monte Carlo) would naturally
quantify parameter uncertainty rather than reporting point estimates.  The
Hessian-based uncertainty ellipsoids computed in Chapter~\ref{ch:hessian}
could serve as an informed prior covariance.

\section{Closing Remarks}

This dissertation demonstrates that Heston calibration for crude oil options is
a genuinely difficult problem, more ill-conditioned than its equity-index
counterpart, and that the degree of ill-conditioning varies predictably with
macro-financial conditions.  The rolling Hessian framework introduced here
provides a reusable diagnostic for any model-market pair, and the negative
correlation between market stress and condition number is an unexpected result
with practical implications for hedgers.

The most immediate practical lesson is one of epistemic humility: a
well-calibrated Heston surface is a useful interpolation device, but the
parameter vector it produces --- particularly the mean-reversion speed $\kappa$
--- should not be taken as a reliable structural estimate.  For applications
that require stable, interpretable parameters (Greeks computation, scenario
analysis, risk limits), a richer model is needed.  Chapter~\ref{ch:svj}
takes precisely this step, applying the same Hessian spectral methodology to
the Bates (1996) SVJ model.  The result --- a condition number of $\approx
2{,}804$ compared with $\approx 52{,}813$ for Heston --- is cautiously
encouraging, though the prevalence of boundary parameters and the negligible
RMSE improvement indicate that further regularisation (e.g.\ Bayesian priors
anchored to historical jump data) is needed before SVJ parameter estimates
can be treated as structurally meaningful.

% ============================================================

% --- Bibliography ---
\bibliographystyle{apalike}
\bibliography{references}

% ============================================================
%  APPENDIX — READING LIST
% ============================================================
\appendix
\chapter{Reading List}
\label{app:reading_list}

The table below consolidates all papers consulted during the preparation of this
dissertation, including the 16 entries from the original reading sheet and 9 additional
references introduced in Chapters~\ref{ch:hessian}, \ref{ch:macro}, and~\ref{ch:svj}.
Entries are numbered continuously; papers added for this dissertation begin at entry~17.

\begin{landscape}
\scriptsize
\setlength{\LTpre}{0pt}
\setlength{\LTpost}{0pt}
\begin{longtable}{
  >{\raggedright\arraybackslash}p{0.4cm}   % #
  >{\raggedright\arraybackslash}p{3.4cm}   % Article
  >{\raggedright\arraybackslash}p{2.0cm}   % Author / Year
  >{\raggedright\arraybackslash}p{2.8cm}   % Objectives
  >{\raggedright\arraybackslash}p{2.8cm}   % Problem Statement
  >{\raggedright\arraybackslash}p{2.8cm}   % Methodology
  >{\raggedright\arraybackslash}p{2.8cm}   % Findings
  >{\raggedright\arraybackslash}p{2.8cm}   % Conclusions
  >{\raggedright\arraybackslash}p{2.6cm}   % Comments / Keywords
}
\toprule
\textbf{\#} &
\textbf{Article} &
\textbf{Author / Year} &
\textbf{Objectives} &
\textbf{Problem Statement} &
\textbf{Methodology} &
\textbf{Findings} &
\textbf{Conclusions} &
\textbf{Comments / Keywords} \\
\midrule
\endfirsthead
\multicolumn{9}{l}{\footnotesize\textit{(continued from previous page)}} \\
\toprule
\textbf{\#} &
\textbf{Article} &
\textbf{Author / Year} &
\textbf{Objectives} &
\textbf{Problem Statement} &
\textbf{Methodology} &
\textbf{Findings} &
\textbf{Conclusions} &
\textbf{Comments / Keywords} \\
\midrule
\endhead
\midrule
\multicolumn{9}{r}{\footnotesize\textit{(continued on next page)}} \\
\endfoot
\bottomrule
\endlastfoot

%--- Entry 1 ---
1 &
The Pricing of Options and Corporate Liabilities &
Black \& Scholes / 1973 &
Price European options on stocks &
Lack of closed-form pricing &
Delta hedge + PDE approach &
Derived BS formula &
Laid foundation for modern option pricing &
GBM / risk-neutral valuation \\
\addlinespace

%--- Entry 2 ---
2 &
The Pricing of Commodity Contracts &
Black / 1976 &
Price options on futures &
Need for futures-based framework &
Modified BS using futures as underlying &
Black-76 model for commodities &
Simplified pricing via forwards &
Black-76 / futures options \\
\addlinespace

%--- Entry 3 ---
3 &
Option Pricing with Jumps &
Merton / 1976 &
Account for jumps in prices &
Lognormal model unrealistic &
Jump-diffusion process &
Model includes jumps in returns &
Improves fit to empirical prices &
Jumps / Poisson process \\
\addlinespace

%--- Entry 4 ---
4 &
Stochastic Behavior of Commodity Prices &
Schwartz / 1997 &
Model commodity spot dynamics &
Non-constant mean-reverting prices &
Two-factor mean-reverting model &
Explains term structure and dynamics &
Supports long-dated option pricing &
Mean reversion / convenience yield \\
\addlinespace

%--- Entry 5 ---
5 &
Evaluating Natural Resource Investments &
Brennan \& Schwartz / 1985 &
Valuing resource projects &
Commodity prices uncertain &
Real options + stochastic yield &
Incorporated convenience yield &
Highlighted real options for commodities &
Stochastic yield / real options \\
\addlinespace

%--- Entry 6 ---
6 &
Options, Futures, and Other Derivatives &
Hull / 2018 &
Comprehensive textbook &
Summarises models and methods &
Pedagogical explanation &
Clarifies model assumptions &
Used for reference and formulas &
Educational / reference \\
\addlinespace

%--- Entry 7 ---
7 &
A Closed-Form Solution for Options with Stochastic Volatility &
Heston / 1993 &
Model with stochastic volatility &
Constant vol fails to explain smiles &
Variance follows CIR process &
Closed-form solution for SV model &
Captures volatility smiles &
Stochastic volatility / CIR \\
\addlinespace

%--- Entry 8 ---
8 &
Jumps and Stochastic Volatility &
Bates / 1996 &
Combine jumps and SV &
Need model for skew \& kurtosis &
Extended Heston + jumps &
Matches observed market prices &
Post-crash implied volatility skew &
SV + jumps / smile fitting \\
\addlinespace

%--- Entry 9 ---
9 &
Oil Option Volatility &
Soini \& Lorentzen / 2019 &
Study crude oil option pricing &
Implied vol deviates from BS &
Empirical analysis of smiles &
Systematic U-shaped vol patterns &
Shows GBM mispricing &
Oil market / smile \\
\addlinespace

%--- Entry 10 ---
10 &
Wheat Futures Under Ukraine War &
Patel / 2024 &
Impact of Ukraine war on wheat &
Why price impact differed by region &
Event study using futures &
US more affected than EU &
Geopolitical shock analysis &
Wheat / geopolitical risk \\
\addlinespace

%--- Entry 11 ---
11 &
Oil Volatility Spike &
US EIA / 2020 &
Document oil volatility surge &
Extreme price swings in 2020 &
Descriptive report &
Oil volatility peaked historically &
Useful for macro regime classification &
Crude / macro regime \\
\addlinespace

%--- Entry 12 ---
12 &
Double Exponential Jump Model &
Kou / 2002 &
Model jumps with skew &
Single-jump models symmetric &
Double exponential jump sizes &
Improves smile fitting &
Asymmetric jumps match data &
Jumps / double exponential \\
\addlinespace

%--- Entry 13 ---
13 &
Nelson-Siegel-Svensson Model &
Rodr\'iguez-S\'anchez / 2024 &
NSS regression &
Collinearity in NS and NSS models &
Regression of yield curves &
Collinearity mitigated &
Collinearity mitigated &
NSS / regressions / yield curve \\
\addlinespace

%--- Entry 14 ---
14 &
Empirical Performance of Alternative Option Pricing Models &
Bakshi, Cao \& Chen / 1997 &
Compare BS, SV, jumps, and SVJ on liquid options &
GBM with constant volatility fails to match smiles and higher moments &
Joint tests of pricing, dynamics, and hedging on cross-sections; panel estimation &
SV and SVJ models reduce pricing and hedging errors versus BS; jumps important &
Richer dynamics materially improve fit; model choice should reflect the use case &
Model comparison / hedging errors / smile / SV / jumps \\
\addlinespace

%--- Entry 15 ---
15 &
Stochastic Convenience Yield and the Pricing of Oil Contingent Claims &
Gibson \& Schwartz / 1990 &
Price oil derivatives with a stochastic convenience-yield factor &
GBM cannot capture oil term structure and storage effects &
Two-factor diffusion (spot and convenience yield); empirical calibration to oil data &
Model explains futures term structure and improves derivative valuation &
Convenience yield is central for energy; single-factor GBM too restrictive &
Commodities / convenience yield / term structure / energy options \\
\addlinespace

%--- Entry 16 ---
16 &
The Importance of the Loss Function in Option Valuation &
Christoffersen \& Jacobs / 2004 &
Show how loss-function selection changes model ranking and inference &
Misalignment between estimation objective and evaluation metric biases conclusions &
Theoretical analysis and empirical comparisons across common loss functions &
Model rankings vary with MAE vs RMSE, price vs vega-normalised errors &
Use a loss function consistent with the study's objective; report sensitivity &
MAE vs RMSE / vega-normalised errors / evaluation design \\
\addlinespace
\midrule
\multicolumn{9}{l}{\textit{Papers added for this dissertation (Chapters 5--6)}} \\
\midrule

%--- Entry 17 ---
17 &
Option Valuation Using the Fast Fourier Transform &
Carr \& Madan / 1999 &
Price options efficiently across many strikes simultaneously &
Analytical characteristic functions are expensive to integrate numerically strike-by-strike &
Damped Fourier transform of the call price; FFT evaluated on a uniform log-strike grid &
One FFT call prices $N = 4{,}096$ strikes; interpolation to arbitrary strikes via cubic spline &
FFT reduces pricing from $O(N^2)$ to $O(N \log N)$; essential for calibration speed &
FFT / characteristic function / Carr-Madan / calibration efficiency \\
\addlinespace

%--- Entry 18 ---
18 &
Complex Logarithms in Heston-Like Models &
Lord \& Kahl / 2010 &
Eliminate numerical instabilities in the Heston characteristic function &
Branch cuts in the complex logarithm cause sign flips and wrong option prices at long maturities &
``Rotation count'' correction that tracks the accumulated argument of the log; applied to Heston and SVJ &
Rotation trick eliminates discontinuities; validated against Monte Carlo benchmarks &
Numerically stable Heston pricing is achievable with a simple branch-cut fix &
Numerical stability / branch cuts / Heston / complex logarithms \\
\addlinespace

%--- Entry 19 ---
19 &
Financial Modelling with Jump Processes &
Cont \& Tankov / 2004 &
Provide a unified mathematical framework for jump process models in finance &
Continuous diffusion models cannot capture the heavy tails, jumps, and skew observed in markets &
L\'evy processes, Poisson random measures, characteristic exponents; calibration and hedging theory &
Comprehensive treatment of pricing, hedging, and calibration for jump models; exponential L\'evy class &
Jump models are necessary for capturing market discontinuities; calibration requires careful regularisation &
L\'evy processes / jump-diffusion / exponential L\'evy / calibration \\
\addlinespace

%--- Entry 20 ---
20 &
Measuring Geopolitical Risk &
Caldara \& Iacoviello / 2022 &
Construct a daily index of geopolitical risk from newspaper text &
No consistent, high-frequency measure of geopolitical risk existed for empirical finance &
Automated text search of major newspaper archives; count of geopolitical-threat articles normalised by total volume &
GPR spikes around wars, terrorist attacks, and geopolitical tensions; significantly affects economic activity and oil prices &
Geopolitical risk is a distinct macroeconomic factor that should be controlled for in commodity and asset pricing &
Geopolitical risk / text analysis / oil prices / macro risk \\
\addlinespace

%--- Entry 21 ---
21 &
Unspanned Stochastic Volatility and the Pricing of Commodity Derivatives &
Trolle \& Schwartz / 2009 &
Model commodity derivatives with volatility factors that cannot be hedged by futures alone &
Standard term-structure models leave large unhedged volatility exposures; futures do not span volatility risk &
Multi-factor affine model with unspanned stochastic volatility; calibration to crude oil futures and options &
Unspanned volatility is quantitatively important for crude oil; futures-only hedges leave material residual risk &
Commodity derivative models require stochastic volatility factors beyond those spanning the futures curve &
Unspanned stochastic volatility / commodity derivatives / crude oil / term structure \\
\addlinespace

%--- Entry 22 ---
22 &
The Volatility Surface: A Practitioner's Guide &
Gatheral / 2006 &
Provide a rigorous practitioner framework for understanding and fitting the implied volatility surface &
Practitioners lack a coherent model-independent framework for the volatility surface &
Survey of stochastic volatility, local volatility, jumps, and their implications for the surface shape &
SVI and SSVI parameterisations; arbitrage constraints on the surface; connection to exotic pricing &
A well-fitted volatility surface requires understanding of model dynamics and surface geometry &
Volatility surface / SVI / SSVI / stochastic vol / local vol / practitioner \\
\addlinespace

%--- Entry 23 ---
23 &
Why are Nonlinear Fits to Data so Challenging? &
Transtrum, Machta \& Sethna / 2010 &
Explain the universal difficulty of nonlinear parameter estimation from an information-geometry perspective &
Optimisers for nonlinear models routinely fail or return highly unstable parameters &
Hessian spectral analysis of the loss landscape; information geometry; stiff-sloppy decomposition &
Model parameter spaces are generically ``sloppy'': a few stiff (well-constrained) directions and many sloppy ones &
Sloppiness is a universal property of complex parametric models and directly explains calibration instability &
Sloppiness / Hessian / parameter identifiability / information geometry / stiff-sloppy \\
\addlinespace

%--- Entry 24 ---
24 &
Exact Simulation of Stochastic Volatility and Other Affine Jump Diffusion Processes &
Broadie \& Kaya / 2006 &
Simulate the Heston model exactly, without Euler discretisation bias &
Standard Monte Carlo for Heston introduces discretisation error; the Feller condition affects simulation stability &
Exact conditional distributions via Laplace transform inversion; exact simulation of the CIR variance process &
Exact simulation eliminates bias at any step size; practically feasible for Heston and AJD processes &
Exact simulation is important for accurate pricing under Feller violations and for validating semi-analytical pricers &
Monte Carlo / exact simulation / Heston / Feller condition / affine jump diffusion \\
\addlinespace

%--- Entry 25 ---
25 &
Jumps and Stochastic Volatility: Exchange Rate Processes Implicit in Deutsche Mark Options &
Bates / 1996 &
Extend Heston SV model to include compound Poisson log-normal jumps; price FX options &
Pure-diffusion Heston cannot account for short-term excess kurtosis and left-tail mass in options prices &
Closed-form characteristic function: $\varphi_\mathrm{SVJ} = \varphi_\mathrm{Heston} \cdot \exp(\lambda T [\phi_J - 1 - \mathrm{iu}\bar{\mu}])$; Carr-Madan FFT pricing &
SVJ fits short-dated implied vol smiles better than pure Heston; jump component absorbs short-term kurtosis; three-parameter extension remains tractable &
SVJ is the natural extension of Heston for commodities with discrete price shocks; used in Chapter~\ref{ch:svj} of this dissertation &
Bates (1996) / SVJ / jump-diffusion / stochastic volatility / characteristic function \\

\end{longtable}
\end{landscape}

\end{document}
